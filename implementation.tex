\chapter{Implementation}

% TODO: explain that objective of implementation is external library
%       that doesn't leak implementation details or require knowledge of monads

This chapter explains the implementation of \ltry\ in Haskell.
It examines how the translations from the previous chapter and the DCC library published in \cite{JonesDS07} are transformed into Haskell code.
Finally, it explores the possibility of a generalized Haskell library.

\section{DCC Library}
% TODO: explain the DCC library, w/ type signatures
% TODO: src in appendix or inline?

\subsection{Missing Typeclass Instances}

\section{Single Naive Implementation}

Blindly transcribing the translation yields a naive implementation in Haskell.
This implementation is a functioning proof-of-concept that the translation works as intended.
It does not present a high level of abstraction, requiring programs to be manually translated to Haskell:

\Verbatimcode
  try :: Prompt ans b 
         -> ((t -> CC ans a) -> CC ans a1) -> (t -> CC ans a1) 
         -> CC ans a1
  try p0 m handler = withSubCont p0 (\n ->
      pushPrompt p0 (pushSubCont n (m $ \x -> 
        throw p0 n handler x)))
  
  throw :: Prompt ans b 
           -> SubCont ans a1 b -> (t -> CC ans a1) -> t 
           -> CC ans a
  throw p0 n c m = withSubCont p0 (\_ ->
    pushPrompt p0 (pushSubCont n (c m)))
\end{Verbatim}

This short implementation:
- adds explicit p0 argument required by wsc and pp.
- requires program to be run inside:

\mono{runCC} with

\begin{figure}[!h]
\Verbatimcode

example1 = runCC (do
  p <- newPrompt
  pushPrompt p (testy p (\t -> return 5) (\x -> return $ x+1)))

example2 = runCC (do
  p <- newPrompt
  pushPrompt p (testy p (\t -> (t 4)) (\x -> return $ x+1)))
\end{Verbatim}
\end{figure}

%%\Verbatim
%%implementation
%%\end{Verbatim}

\subsection{Issues}
Remark: wrap throw in a function to solve scoping issues

Impedance mismatch between DCC and \ltry-calculus. 
We only need a single prompt. 
This prompt should be triggered on beginning of "try" statement. 
Prompt should be reused for setting up all catch handlers statements

Access to each handler requires explicitly passing catch names to solve scoping.
This means that we have to define try, try1, try2, etc for number of handlers:
no approach for variadic handlers.

Type signature exposes CC, Prompt, etc data constructors.
We want to abstract over these so user does not rely on knowing them.
Leaky abstractions.

\section{Improved Implementation}

The improved version creates a single prompt for the try block.
The throw operation is defined in scope of the prompt, 
removing the need to add an additional argument.

We are still required to call runCC to return the result.
Although handlers can be ordinary functions (see examples), 
the body must \mono{return} a value if not calling one of the exception handlers.
Also, multiple handlers still require multiple function instances:

\Verbatimcode
try m handler = do
  p <- newPrompt
  let throw n = \v -> withSubCont p (\_ ->
            pushPrompt p (pushSubCont n $ return v))
  pushPrompt p (withSubCont p (\n ->
    pushPrompt p (pushSubCont n 
      (m (thr n . handler)))))
        
try2 m h1 h2 = do
  p <- newPrompt
  let throw n = \v -> withSubCont p (\_ ->
            pushPrompt p (pushSubCont n $ return v))
  pushPrompt p (withSubCont p (\n ->
    pushPrompt p (pushSubCont n 
      (m (thr n . h1) (thr n . h2)))))
\end{Verbatim}

\Verbatimcode
handler1 = (+2)
handler2 = (+4)

tryExample1 = try (\t -> return 1) handler1
tryExample2 = try (\t -> t 1) handler1

try2Example1 = try2 (\t1 -> \t2 -> return 1) handler1 handler2
try2Example2 = try2 (\t1 -> \t2 -> t1 1) handler1 handler2
try2Example3 = try2 (\t1 -> \t2 -> t2 1) handler1 handler2
\end{Verbatim}

If we look at the type signatures, we can see the same important restrictions from the naive implementation.
The handlers must produce values of the same type. 
This will also be the type of running \mono{runCC} on the returned \mono{CC} value.


\Verbatimcode
  try  :: ((a -> CC ans a1) -> CC ans a2) -> (a -> a2) -> CC ans a2
  try2 :: ((a -> CC ans a1) -> (a2 -> CC ans a3) -> CC ans a4)
           -> (a -> a4) -> (a2 -> a4) -> CC ans a4
\end{Verbatim}

