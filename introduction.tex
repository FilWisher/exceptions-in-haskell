\chapter{Introduction}

Explanation of problem space: need and motivation demonstrated with examples.

What are exceptions? How are they typed? What have approaches been before?

van Bakel and the \ltry-calculus is different approach. \ltry\ already compared
to the 'classical- cal...

Exceptions have been done but unnamed or dispatch on type.
\ltry\ introduces exceptions with names.

Features of computer programs are discovered twice: by logicians and by computer scientists.\cite{Wadler15}
Exceptions have been mapped to continuations which have been mapped to classical logic.
What about a calculus that models exceptions directly?
HOw does it behave?

\section{Solution}
Use van Bakel's translation of \ltry\ to \lmu.
Define a translation from \lmu\ to CDC, which closely models Haskell's syntax.
Write a CDC interpreter for generating derivations that can be transcribed into proofs.
Investigate properties of \lmu\ translation.
Use this translation to find a translation from \ltry\ to lmu.
Implement \ltry\ directly in Haskell by following this.

\section{Contributions}
This paper makes the following contributions:
\begin{itemize}
\item Haskell interpreter for a calculus of delimited continuations, CDC, written by SPJ
\item Translation of \lmu\ to CDC along with proof of soundness and completeness with respect to mu reduction.
\item A translation of \ltry\ to CDC.
\item A proof of concept implementation of \ltry\ in Haskell, based on this translation.
\item The specification for a language extension for named exceptions in Haskell, based on \ltry.
\end{itemize}
