\chapter{Introduction}

In 1936 Alonzo Church presented the \lam-calculus, a formal system for expressing the entire set of computable problems \cite{Church36}.\footnote{
  The ability of a system to express the entire set of computable problems is now commonly known as `Turing completeness' although here this description would be anachronistic; 
  Turing's work on computability was published a year after Church's.
}
Later, Haskell Curry and William Howard both discover a correspondence between the simply-typed \lam-calculus and intuitionistic logic. 
However the correspondence runs deeper than this: there is actually a true isomorphism between many systems of formal logic and computational calculi. 

The correspondence between systems of formal logic and computer programs drove research into both the logical counterparts of programs and the computational counterparts of logical systems.
As a part of this, Michel Parigot developed the \lmu-calculus, an extension of the \lam-calculus, 
through an isomorphism to classical logic \cite{Parigot92}.
More recently, Steffen van Bakel modelled exceptions and exception handling through an extension of the \lam-calculus called the \ltry-calculus \cite{Bakel15}.\footnote{
  Exceptions are a form of error generated by computers to indicate that the flow of control cannot continue as expected and must be aborted. 
}

Exceptions are most commonly found in two kinds: 
  systems where there is a single type of exception such as in Javascript, 
  and systems where exceptions are discriminated by \emph{type} as in Java and Haskell. 
Unlike either of these approaches, 
the exception handling proposed by the \ltry-calculus introduces exceptions discriminated by \emph{name}.

\subsubsection{Project}
This paper proposes an implementation of named exceptions in Haskell.
First we define a translation from the \lmu-calculus to a calculus for expressing delimited continuations, CDC \cite{JonesDS07}.
\[
\begin{array}{lcr}
  `l`m & \to & CDC
\end{array}
\]
CDC is a calculus which closely models Haskell's syntax and which has been extended with operations for manipulating delimited continuations.
The CDC library for Haskell defined by Jones \emph{et al.} in \cite{JonesDS07} evaluates CDC terms in a single step, returning the final value of the reduction.
However, we are interested in producing the entire step-by-step evaluation of CDC terms which we can transliterate into CDC derivations for verifying translations of \lmu\ into CDC.
This motivates us to write a term-rewriting program for evaluating CDC terms.
The proofs generated by this program are used to investigate the soundness and completeness of the $`l`m \to CDC$ translation.
The translation is then concatenated with van Bakel's original translation of his \ltry-calculus into the \lmu-calculus which yields a translation from \ltry\ to CDC:
\[
   (`l^{try} \to `l`m) \circ (`l`m \to CDC) = `l^{try} \to CDC
\]
We use this final translation to explore implementations of named exceptions directly in Haskell.

\subsubsection{Outline}
Chapter \ref{chapter:background} discusses the related context for this project. Following this, this paper makes the following original contributions:
\begin{itemize}
\item A Haskell interpreter for the calculus of delimited continuations, CDC (in Chapter \ref{chapter:interpreter})
\item A translation of \lmu\ into CDC along with proofs of its soundness and completeness with respect to $`m$-reduction (in Chapter \ref{chapter:translations})
\item A translation of \ltry\ into CDC (also in Chapter \ref{chapter:translations}) 
\item A proof of concept implementation of \ltry\ in Haskell, based on this translation (in Chapter \ref{chapter:implementation})
\item The specification for a language extension for Haskell with named exceptions, based on \ltry\ (also Chapter \ref{chapter:implementation})
\end{itemize}
