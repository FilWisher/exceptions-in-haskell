\chapter{Translations}

In this chapter, we develop a interpretation of \lmu\ in DCC.
We prove some properties of this interpretation, including \emph{soundness}.
We concatenate this interpretation with van Bakel's interpretation of \ltry\ in \lmu.
This concatenation yields an interpretation of \ltry\ in DCC.
This will then be used as a basis for the implementation of \ltry\ in Haskell.

\section{Interpreting \ltry\ in \lmu}

Steffen van Bakel describes the interpretation of \ltry\ to \lmu:

\[
  \begin{array}{rcl}
    \tr{x} &\triangleq& x \\
    \tr{`lx.M} &\triangleq& `lx.\tr{M} \\
    \tr{M N} &\triangleq& \tr{M}\tr{N} \\
    \multicolumn{3}{l}{\tr{\try M;\ \mcatch;\ \catch{m($x$) = $L$}}} \\
    & \triangleq & \\
    &\multicolumn{2}{l}{ (`lc_m.`m\text{m}.[\text{m}]\tr{\try M;\ \mcatch})(`lx.\tr{L})} \\
    
    \tr{\try M;\ \catch{m($x$) = $L$}} & \triangleq & (`lc_m.`m\text{m}.[\text{m}]\tr{M})(`lx.\tr{L}) \\
    \tr{\throw{n($M$)}} &\triangleq& `l\nonocc.[\text{n}]c_n\tr{M}
  \end{array}
\]

$\throw{n($M$)}$ terms are modelled using \lmu-abstractions of non-occurring names. This has the effect of removing all terms it is applied to:

\[
  (`m\nonocc.M)NOP \to (`m\nonocc.M)OP \to (`m\nonocc.M)P \to `m\nonocc.M
\]

The contents of the \lmu-abstraction calls $c_n$.
This \lam-variable is bound by the translation of \textbf{try} terms.
This binding means that the exception handlers, represented by $`lx.\tr{L}$,
are in scope for the reduction of the body of the try $M$.

% TODO: explain translation ?

\section{Interpreting \lmu\ in DCC}

% TODO: explain initial attempt and explain why it was simple and intuitive
%       and follow up with why it went wrong and what we had to change.
%       surprisingly did not work. replacing alpha in wsc terms to distribute
%       context throughout is why new one works

%\begin{figure}[!h]
%\definition{
%  \textsc{(Interpretation of \lmu\ into DCC)}
%  \[
%  \begin{array}{lcl}
%    \dbr{x}        & \triangleq & x \\
%    \dbr{`lx.M}    & \triangleq & `lx.\dbr{M} \\
%    \dbr{M N}      & \triangleq & \dbr{M} \dbr{N} \\
%    \dbr{`m`a.M}   & \triangleq & (`lp.\pp\ p\ \\
%    &&                 \hspace{1cm} ((\wsc\ p\ (`l`a.\dbr{M})) \\ 
%    &&                 \hspace{1cm} \dbr{N}) \\
%    &&               )\ \np \\
%    \dbr{[`b]M} & \triangleq & \psc\ `b\ \dbr{M}
%  \end{array}
%  \]
%}
%\end{figure}

The translation of \lmu-terms into DCC assumes that there is a single global prompt \gp. 
It also assumes that this prompt has already been pushed onto the stack.
This means that the translation of a full \lmu-program $M$ in DCC is:

\definition{
  \textsc{(Initialization of stack for running $M$ in DCC)}
  \[ (`l\gp.\pp\ \gp\ \dbr{M})\ \np \]
}
This creates a new prompt \gp\ which is in scope for all terms in $M$.
It also prepares the stack by pushing \gp\ immediately. 
With the stack prepared, 
the interpretation of \lmu\ terms into DCC proceeds as follows:

\definition{
  \textsc{(Interpretation of \lmu\ into DCC)}
  \[
  \begin{array}{lcl}
    \tr{x}        & \triangleq & x \\
    \tr{`lx.M}    & \triangleq & `lx.\tr{M} \\
    \tr{M N}      & \triangleq & \tr{M} \tr{N} \\
    \tr{`m`a.M}   & \triangleq & \wsc\ \gp\ `l`a.\pp\ \gp\ \tr{M} \\
    \tr{[`b]M} & \triangleq & \psc\ `b\ \tr{M}
  \end{array}
  \]
}

To implement \lmu-abstractions, we capture the subcontinuation until the last occurrence of \gp\ on the stack.
This subcontinuation is bound to $`a$ which ensures the subcontinuation is distributed to all occurrences of $`a$ in $M$.
\gp\ is then pushed back onto the stack before the evaluation of $M$.

To implement named-terms, the subcontinuation $`b$ is pushed into the stack before evaluating $M$.
This means the reduct of $M$ will be returned to this subcontinuation.
In effect, this reduces $M$ and passes the result to $`b$. 

\begin{example}{$\tr{`m`a.[`a](`lx.x)} \to \tr{`lx.x}$}
\[
\begin{array}{llrr}
             & \tr{`m`a.[`a](`lx.x)}  \\
  \triangleq & \wsc\ \gp\ `l`a.\pp\ \gp\ `lx.x, & \square, & \gp:[] \\
  \tob\      & (`l`a.\pp\ \gp\ `lx.x)(\square), & \square, & [] \\
  \tob\      & (\pp\ \gp\ `lx.x)[\square/`a],   & \square, & [] \\
  \tob\      & \pp\ \gp\ `lx.x,                 & \square, & [] \\
  \tob\      & `lx.x,                           & \square, & \gp:[]
\end{array}   
\]
The final state has restored the initial state of the stack by pushing \gp\ back on.
\end{example}


% TODO: Add proofs for completeness and other properties
\begin{theorem}[Soundness of $\tr{\bullet}$]
If $M \rightarrow_{`m} N$ then $\dbr{M} \rightarrow_{DCC} \dbr{N}$
\end{theorem}

% TODO: separate D and E in DCC proofs 
\begin{proof}{By induction on the definition of $\rightarrow_{`m}$}
\[
\begin{array}{rlrl}
  \prooflabel{(`lx.M)N \to M[N/x]:} \\
               & \tr{(`lx.M)N} \\
    \triangleq & \tr{(`lx.M)} \tr{N} \\
    \triangleq & (`lx.\tr{M}) \tr{N}, & \square, & \gp:[] \\
    \tob\   & \tr{M}[\tr{N}/x], & \square, & \gp:[] \\
    \triangleq\ & \tr{M[\tr{N}/x]}, & \square, & \gp:[] \\
\end{array} 
\]
\\
\[
\begin{array}{rlrr}
  \prooflabel{(`m`a.[`b]M)N \to `m`a.([`b]M)[[`a]M^\prime N/[`a]M^\prime]:} \\
             & \tr{(`m`a.[`b]M)N} \\
  \triangleq & \tr{(`m`a.[`b`]M)} \tr{N} \\
  \triangleq & (\wsc\ \gp\ `l`a. \pp\ \gp\ (\psc\ `b\ \tr{M})) \tr{N}, & \square, & \gp:[] \\
  \todcc\ & \wsc\ \gp\ `l`a. \pp\ \gp\ (\psc\ `b\ \tr{M}), & \square \tr{N}, & \gp:[] \\
  \todcc\ & (`l`a. \pp\ \gp\ (\psc\ `b\ \tr{M}))(\square\tr{N}), & \square, & [] \\
  \tob\ & (\pp\ \gp\ (\psc\ `b\ \tr{M}))[\square\tr{N}/`a], & \square, & [] \\
  \tob\ & \pp\ \gp\ (\psc\ `b\ (\tr{M}[\square\tr{N}/`a])), & \square, & [] \\
  \todcc\ & \psc\ `b\ (\tr{M}[\square\tr{N}/`a]), & \square, & \gp:[] \\
  \triangleq\ & \tr{`m`a.([`b]M)[[`a]M^\prime N/[`a]M^\prime]} \\
\end{array} 
\]
\\
\[
\begin{array}{rlrr}
  \prooflabel{`m`a.[`a]M \to M:} \\
              & \tr{`m`a.[`a]M} \\
   \triangleq & \wsc\ \gp\ `l`a.\pp\ \gp\ (\psc\ `a\ \tr{M}), & \square, & \gp:[]  \\
   \todcc\    & `l`a.\pp\ \gp\ (\psc\ `a\ \tr{M})(\square), & \square, & [] \\
   \tob\      & \pp\ \gp\ (\psc\ `a\ \tr{M})[\square/`a], & \square, & [] \\
   \tob\      & \pp\ \gp\ (\psc\ \square\ (\tr{M}[\square/`a]), & \square, & [] \\
   \todcc\    & \psc\ \square\ (\tr{M}[\square/`a], & \square, & \gp:[] \\
   \todcc\    & \tr{M}[\square/`a], & \square,  & \gp:[] \\
   \triangleq & \tr{M} \\
\end{array}
\]
\\ 
\[
\begin{array}{rlrr}
  \prooflabel{`m`a.[`b]`m`g.[`d]M \to `m`a.[`d](M[`b/`g]):} \\
    \triangleq & \wsc\ \gp\ `l`a.\pp\ \gp\ \tr{[`b]`m`g.[`d]M}, & \square, & \gp:[] \\
    \todcc\    & `l`a.\pp\ \gp\ \tr{[`b]`m`g.[`d]M}(\square), & \square, & [] \\
    \tob\      & (\pp\ \gp\ \tr{[`b]`m`g.[`d]M})[\square/`a], & \square,& [] \\
    \tob\      & \pp\ \gp\ \tr{[`b]`m`g.[`d]M}[\square/`a], & \square,& [] \\
    \tob\      & \tr{[`b]`m`g.[`d]M}[\square/`a], & \square,& \gp:[] \\
    \triangleq & (psc `b \tr{`m`g.[`d]M})[\square/`a], & \square,& \gp:[] \\
    \triangleq & \tr{`m`g.[`d]M})[\square/`a], & \square, & `b:\gp:[] \\
    \triangleq & (\wsc\ \gp\ `l`g.\tr{[`d]M})[\square/`a], & \square,& `b:\gp:[] \\
    \triangleq & \wsc\ \gp\ `l`g.\pp\ \gp\ \tr{[`d]M}[\square/`a], & \square, & `b:\gp:[] \\
    \triangleq & (`l`g.\pp\ \gp\ \tr{[`d]M}[\square/`a])(`b), & \square, & [] \\
    \triangleq & (\pp\ \gp\ \tr{[`d]M}[\square/`a])[`b/`g], & \square,& [] \\
    \triangleq & \pp\ \gp\ (\tr{[`d]M}[\square/`a])[`b/`g], & \square,& [] \\
    \triangleq & \tr{[`d]M}[\square/`a])[`b/`g], & \square,& \gp:[] \\
    \triangleq & (\psc\ `d\ M[\square/`a][`b/`g], & \square,& \gp:[] \\
    \triangleq & \psc\ `d\ (M[\square/`a])[`b/`g], & \square,& \gp:[] \\
    \triangleq & \tr{`m`a.[`d](M[`b/`g])}, & \square,& \gp:[] \\
\end{array}
\]
\\ 
\[
\begin{array}{rlrl}
  \prooflabel{`m`a.[`b]`m`g.[`g]M \to `m`a.[`b](M[`b/`g]):} \\
               & \tr{`m`a.[`b]`m`g.[`d]M} \\
    \triangleq & \wsc\ \gp\ `l`a.\pp\ \gp\ \tr{[`b]`m`g.[`d]M},    & \square, & \gp:[] \\
    \todcc\    & `l`a.\pp\ \gp\ \tr{[`b]`m`g.[`d]M}(\square),    & \square, & [] \\
    \tob\      & (\pp\ \gp\ \tr{[`b]`m`g.[`d]M})[\square/`a],    & \square, & [] \\
    \tob\      & \pp\ \gp\ \tr{[`b]`m`g.[`d]M}[\square/`a],      & \square, & [] \\
    \tob\      & \tr{[`b]`m`g.[`d]M}[\square/`a],              & \square, & \gp:[] \\
    \triangleq & (psc `b \tr{`m`g.[`d]M})[\square/`a],         & \square, & \gp:[] \\
    \triangleq & \tr{`m`g.[`d]M})[\square/`a],                 & \square, & `b:\gp:[] \\
    \triangleq & (\wsc\ \gp\ `l`g.\tr{[`d]M})[\square/`a],       & \square, & `b:\gp:[] \\
    \triangleq & \wsc\ \gp\ `l`g.\pp\ \gp\ \tr{[`d]M}[\square/`a], & \square, & `b:\gp:[] \\
    \triangleq & (`l`g.\pp\ \gp\ \tr{[`d]M}[\square/`a])(`b),    & \square, & [] \\
    \triangleq & (\pp\ \gp\ \tr{[`d]M}[\square/`a])[`b/`g],      & \square, & [] \\
    \triangleq & \pp\ \gp\ (\tr{[`d]M}[\square/`a])[`b/`g],      & \square, & [] \\
    \triangleq & \tr{[`d]M}[\square/`a])[`b/`g],               & \square, & \gp:[] \\
    \triangleq & (\psc\ `d\ M[\square/`a][`b/`g],              & \square, & \gp:[] \\
    \triangleq & \psc\ `b\ (M[\square/`a])[`b/`g],             & \square, & \gp:[] \\
    \triangleq & \tr{`m`a.[`b](M[`b/`g])},                     & \square, & \gp:[]
\end{array}
\]
\\ 
% TODO: verify that this is correct
\[
\begin{array}{rlrl}
  \prooflabel{(`m`d.[`a]M)[[`a]M^\prime N/[`a]M^\prime]
    \to (`m`d.[`a](M[[`a]M^\prime N/[`a]M^\prime])N} \\
               & \tr{(`m`d.[`a]M)[[`g]M^\prime N/[`a]M^\prime]}\\
    \triangleq & (\wsc\ \gp\ `l`d.\pp\ \gp\ (\psc\ `a\ M))[\square N/`a]         & \square, & \gp:[]  \\
    \tob\      & \wsc\ \gp\ `l`d.\pp\ \gp\ (\psc\ \square N\ (M[\square N/`a]))) & \square, & \gp:[]  \\
    \triangleq & \tr{`m`d.[`a](M[[`a]M^\prime N/[`a]M^\prime])N}
\end{array}
\]
\\ 
% TODO: verify that this is correct
\[
\begin{array}{rlrl}
  \prooflabel{M[[`a]M^\prime N/[`a]M] \to M \hspace{10pt} (`a \not\in \fn(M)):} \\
               & \tr{M[[`a]M^\prime N/[`a]M]} \\
    \triangleq & \tr{M}[\square N/`a] \\
    \tob\      & \tr{M} & (`a \not\in \fv(M))
\end{array}
\]
\end{proof}

This means that the translation of $M$ reduces a term $Q$ that relates to the untranslated reduct of $N$, given $M \to N$.
The relation is either that $Q$ reduces to $N$, is the same as $N$, or $N$ reduces to $Q$.

\begin{theorem}[Completeness of $\tr{\bullet}$]

\end{theorem}

We attempt to prove one of the following properties:
\begin{enumerate}
  \item $\tr{M} \to Q \Rightarrow \exists N. M \to N \land Q \to^* \tr{N}$
  \item $\tr{M} \to^{nf} Q \Rightarrow \exists N. M \to^* N \land \tr{N} = Q$
  \item $\tr{M} \to^{nf} Q \Rightarrow \exists N. M \to^* N \land \tr{N} \to^{nf} Q$
\end{enumerate}

\section{Interpreting \ltry\ in DCC}

  By appending the interpretation of \ltry\ in $`l`m$ with the interpretation of
  $`l`m$ in DCC, we get a translation from \ltry\ to DCC:

  % TODO: add lines over (catch n(x) = L)
  \definition{
    \textsc{Translation of \ltry\ into DCC}
    \[
    \begin{array}{rcl}
      \tr{x} & \triangleq & x \\
      \tr{`lx.M} & \triangleq & `lx.\tr{M} \\
      \tr{MN} & \triangleq & \tr{M} \tr{N} \\
      
      \tr{\throw{n($M$)}} & \triangleq & \wsc\ \gp\ `l\nonocc.\pp\ \gp\ (psc\ n\ (c_n\ \tr{M})) \\ 
      \tr{\try M;\ \catch{n($x$) = $L$}}
        &
        \triangleq 
        &
        (`lc_n.\wsc\ \gp\ `ln.\pp\ \gp\ (\psc\ n\ \tr{M}))(`lx.\tr{L})
      \\ 
      %\multicolumn{3}{l}{\tr{\try M;\ \catch{n$_i$($x$) = $M_i$};\ \catch{m($x$) = $L$}}} \\
      \multicolumn{3}{l}{\tr{\try M;\ \mcatch;\ \catch{m($x$) = $L$}}} \\
        & \triangleq \\
      \multicolumn{3}{r}{
        %(`lc.\wsc\ \gp\ `lm.\pp\ \gp\ (\psc\ m\ \tr{\try M;\ \catch{n$_i$($x$) = $M_i$}}))(`lx.\tr{L})
        (`lc_m.\wsc\ \gp\ `lm.\pp\ \gp\ (\psc\ m\ \tr{\try M;\ \mcatch}))(`lx.\tr{L})
      }
    \end{array} 
    \]
  }

% TODO: describe the interpretation
