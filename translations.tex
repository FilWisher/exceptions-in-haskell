\chapter{Translations}

% TODO: explain nf reduction in background

In this chapter, we develop a interpretation of \lmu\ in CDC.
We prove some properties of this interpretation, including \emph{soundness}.
We concatenate this interpretation with van Bakel's interpretation of \ltry\ in \lmu.
This concatenation yields an interpretation of \ltry\ in CDC.
This will then be used as a basis for the implementation of \ltry\ in Haskell.

\section{Interpreting \ltry\ in \lmu}

van Bakel describes the interpretation of \ltry\ to \lmu:

{
\relscale{0.9}
\[
  \begin{array}{rcl}
    \tr{x} &\triangleq& x \\
    \tr{`lx.M} &\triangleq& `lx.\tr{M} \\
    \tr{M N} &\triangleq& \tr{M}\tr{N} \\
    \multicolumn{3}{l}{\tr{\try M;\ \mcatch;\ \catch{m($x$) = $L$}}} \\
    & \triangleq & \\
    &\multicolumn{2}{l}{ (`lc_m.`m\text{m}.[\text{m}]\tr{\try M;\ \mcatch})(`lx.\tr{L})} \\
    
    \tr{\try M;\ \catch{m($x$) = $L$}} & \triangleq & (`lc_m.`m\text{m}.[\text{m}]\tr{M})(`lx.\tr{L}) \\
    \tr{\throw{n($M$)}} &\triangleq& `l\nonocc.[\text{n}]c_n\tr{M}
  \end{array}
\]
}

$\throw{n($M$)}$ terms are modelled using \lmu-abstractions of non-occurring names. This has the effect of removing all terms it is applied to:

\[
  (`m\nonocc.M)NOP \to (`m\nonocc.M)OP \to (`m\nonocc.M)P \to `m\nonocc.M
\]

The contents of the \lmu-abstraction calls $c_n$.
This \lam-variable is bound by the translation of \textbf{try} terms.
This binding means that the exception handlers, represented by $`lx.\tr{L}$,
are in scope for the reduction of the body of the try $M$.

% TODO: explain translation ?

\section{Interpreting \lmu\ in CDC}\label{sec:lmu-in-cdc}

% TODO: explain initial attempt and explain why it was simple and intuitive
%       and follow up with why it went wrong and what we had to change.
%       surprisingly did not work. replacing alpha in wsc terms to distribute
%       context throughout is why new one works

%\begin{figure}[!h]
%\definition{
%  \textsc{(Interpretation of \lmu\ into CDC)}
%  \[
%  \begin{array}{lcl}
%    \dbr{x}        & \triangleq & x \\
%    \dbr{`lx.M}    & \triangleq & `lx.\dbr{M} \\
%    \dbr{M N}      & \triangleq & \dbr{M} \dbr{N} \\
%    \dbr{`m`a.M}   & \triangleq & (`lp.\pp\ p\ \\
%    &&                 \hspace{1cm} ((\wsc\ p\ (`l`a.\dbr{M})) \\ 
%    &&                 \hspace{1cm} \dbr{N}) \\
%    &&               )\ \np \\
%    \dbr{[`b]M} & \triangleq & \psc\ `b\ \dbr{M}
%  \end{array}
%  \]
%}
%\end{figure}

The translation of \lmu-terms into CDC assumes that there is a single global prompt \gp. 
It also assumes that this prompt has already been pushed onto the stack.
This means that the translation of a full \lmu-program $M$ in CDC is:

\definition{
  \textsc{(Initialization of stack for running $M$ in CDC)}
  \[ (`l\gp.\pp\ \gp\ \dbr{M})\ \np \]
}
This creates a new prompt \gp which is in scope for all subterms of $M$.
It also prepares the stack by pushing \gp immediately. 
With the abstract machine prepared, 
the interpretation of \lmu\ terms into CDC proceeds as follows:

\definition{
  \textsc{(Interpretation of \lmu\ into CDC)}
  \[
  \begin{array}{lcl}
    \tr{x}        & \triangleq & x \\
    \tr{`lx.M}    & \triangleq & `lx.\tr{M} \\
    \tr{M N}      & \triangleq & \tr{M} \tr{N} \\
    \tr{`m`a.M}   & \triangleq & \wsc\ \gp\ (`l`a.\pp\ \gp\ \tr{M}) \\
    \tr{[`b]M} & \triangleq & \psc\ `b\ \tr{M}
  \end{array}
  \]
}

To implement $`m$-abstractions, we capture the subcontinuation until the last occurrence of \gp on the stack.
This subcontinuation is bound to $`a$ which ensures the subcontinuation is distributed to all occurrences of $`a$ in $M$.
\gp is then pushed back onto the stack before the evaluation of $M$.

To implement named-terms, the subcontinuation $`b$ is pushed into the stack before evaluating $M$.
This means the reduct of $M$ will be returned to this subcontinuation.
In effect, this will reduce $M$ and passes the result to $`b$.

\subsection{Notation}
To carry out proofs, the full state of the abstract machine is displayed:
\[
\begin{array}{lll}
  M & D & E
\end{array}
\]
where each column corresponds to one component of the original abstract machine.
Our translations only use a single prompt so we omit the final column of the abstract machine 
(used for representing the global prompt counter).

When an empty context is in a sequence, it has no effect on the machine:
a sequence $D:\square:D^\prime$ is extensionally equivalent to $D:D^\prime$.
For this reason, we omit empty contexts from sequences.
For example in the case of the following reduction
\[
\begin{array}{lllclll}
  \psc `b\ M & \square & \gp & \tocdc\ & M & \square & `b:\square:\gp
\end{array}
\]
we will instead write
\[
\begin{array}{lllclll}
  \psc `b\ M & \square & \gp & \tocdc\ & M & \square & `b:\gp
\end{array}
\]


\subsection{Additional Translations}
We alter $`m$-reduction to consume multiple variables.
The application of a $`m$ abstraction to multiple variables will consume them all at once:
\definition{
\textsc{$`m$-reduction to consume multiple variables}
\[ 
\begin{array}{rcl}
  (`m`a.[`b]M)\mult{N} & \to_{`m} & `m`a.([`b]M\{[`a]M'\mult{N}/[`a]M'\})
\end{array}
\]
}
This does not change the behaviour of $`m$-reduction but condenses the reduction steps.
The entire applicative context is consumed.
Therefore the remaining $`m$ abstraction will point $`a$ to $\square$.
This means that all labelled subterms $[`a]M'$ will be translated to $\psc \square\ \tr{M'}$.
CDC reduces $\psc \square\ \tr{M'}$ to $\tr{M'}$.
This reduction means we can discard the $`a$ labels after consuming the entire context.
Given this, we can define the following translation for multiple-variable consumption:
\definition{\label{def:multiple-variable-consumption}
\textsc{Translation of multiple variable consumption to CDC}
\[ 
\begin{array}{rcl}
  \tr{M[[`a]M'\mult{N}/[`a]M']} & \triangleq & \tr{M}[\square \tr{\mult{N}}/`a]
\end{array}
\]
}

\begin{example}{Example of multiple variable consumption in CDC and \lmu}
\[
\begin{array}{lcrr}
         & (`m`a.[`a](`lx.`ly.xy))st \\
\to_{`m} & `m`a.([`a](`lx.`ly.xy))\{[`a]Mst/[`a]Mst\}) \\
\end{array}
\]
From here we either can continue with the \lmu\ reduction:
\[
\begin{array}{lcrr}
         & `m`a.([`a](`lx.`ly.xy))\{[`a]Mst/[`a]Mst\}) \\
\to_{`m} & `m`a.[`a](`lx.`ly.xy)st \\
\to_{`b} & `m`a.[`a](`ly.sy)t \\
\to_{`b} & `m`a.[`a]st \\
\to_{`m} & st & (`a \notin fn(st)) \\
\end{array}
\]
We can also continue by translating to and reducing in CDC:
{
\relscale{0.9}
\[
\begin{array}{lcrr}
           & \tr{`m`a.([`a](`lx.`ly.xy))\{[`a]Mst/[`a]Mst)\}} \\
\triangleq & \wsc \gp (`l`a.\pp \gp \tr{([`a](`lx.`ly.xy))\{[`a]Mst/[`a]Mst\})}) & \square & \gp \\
\to        & (`l`a.\pp \gp \tr{([`a](`lx.`ly.xy))\{[`a]Mst/[`a]Mst\})})(\square) & \square & [] \\
\to_{`b}   & (\pp \gp \tr{([`a](`lx.`ly.xy))\{[`a]Mst/[`a]Mst\})})[\square/`a] & \square & [] \\
\to        & \pp \gp \tr{([`a](`lx.`ly.xy))\{[`a]Mst/[`a]Mst\})}[\square/`a] & \square & [] \\
\to        & \tr{([`a](`lx.`ly.xy))\{[`a]Mst/[`a]Mst\})}[\square/`a] & \square & \gp \\
\to        & ((\psc `a (`lx.`ly.xy))[\square st/`a])[\square/`a] & \square & \gp \\
\to        & (\psc \square st\ (`lx.`ly.xy))[\square/`a] & \square & \gp \\
\to        & \psc \square st\ (`lx.`ly.xy) & \square & \gp \\
\to        & (`lx.`ly.xy)   & \square & \square st:\gp \\
\to        & (`lx.`ly.xy)   & \square st\ & \gp \\
\to        & (`lx.`ly.xy)st & \square & \gp \\
\to_{`b}   & (`ly.sy)t & \square & \gp \\
\to_{`b}   & st & \square & \gp \\
\end{array}
\]
}
\end{example}

Using the translation in Definition \ref{def:multiple-variable-consumption}, the proof that $\tr{`.}$ respects $\to_{`m}$ follows easily.

\subsection{Properties}

% TODO: Add proofs for completeness and other properties
\begin{theorem}[Soundness of $\tr{`.}$]\label{theorem:soundness}
\[ M \to_{`m} N \Rightarrow \exists P.\ \tr{M} \to^{*} P \land \tr{N} \to^{*} P \]
\end{theorem}

% TODO: remark: mu abstractions can only occur
% TODO: separate D and E in CDC proofs 
\begin{Proof}
By induction on the definition of $\rightarrow_{`m}$
\[
\begin{array}{@{}lcrr}
  \prooflabel{`m`a.[`a]M \to M} & (`a \notin fn(M)) \\
             & \tr{`m`a.[`a]M} \\
  \triangleq & \wsc \gp `l`a.\pp \gp (\psc `a \tr{M}) & D       & \gp \\
  \tocdc\    & (`l`a.\pp \gp (\psc `a \tr{M}))(D)     & \square & []     \\
  \tocdc\    & (\pp \gp (\psc `a\ \tr{M}))[D/`a]      & \square & []     \\
  \tocdc\    & \pp \gp (\psc `a\ \tr{M})[D/`a]        & \square & []     \\
  \tocdc\    & (\psc `a\ \tr{M})[D/`a]                & \square & \gp \\
  \tocdc\    & \psc D\ \tr{M}[D/`a]                   & \square & \gp \\
  \tocdc\    & \tr{M}[\square/`a]                     & \square & D:\gp \\
  \tocdc\    & \tr{M}                                 & \square & D:\gp \\
  \tocdc\    & \tr{M}                                 & D       & \gp \\
  \triangleq & M                                      & D       & \gp \\
\end{array}
\]
\[
\begin{array}{@{}lcrr}
  \prooflabel{(`m`a.[`a]M)\mult{N} \to `m`a.[`a]M\{[`a]M'\mult{N}/[`a]M'\}\mult{N}} & (`a \notin fn(\mult{N})) \\
             & \tr{(`m`a.[`a]M)\mult{N}} \\
  \triangleq & \tr{(`m`a.[`a]M)}\tr{\mult{N}} \\
  \triangleq & (\wsc \gp `l`a.\pp \gp (\psc `a\ \tr{M})) \tr{\mult{N}} & \square                & \gp \\
  \tocdc\    & \wsc \gp `l`a.\pp \gp (\psc `a\ \tr{M})                 & \square \tr{\mult{N}}  & \gp \\
  \tocdc\    & `l`a.\pp \gp (\psc `a\ \tr{M})(\square \tr{\mult{N}})   & \square                & [] \\
  \tocdc\    & (\pp \gp (\psc `a\ \tr{M}))[\square \tr{\mult{N}}/`a]   & \square                & [] \\
  \tocdc\    & \pp \gp (\psc `a\ \tr{M})[\square \tr{\mult{N}}/`a]     & \square                & [] \\
  \tocdc\    & \pp \gp (\psc \square \tr{\mult{N}}\ \tr{M}[\square \tr{\mult{N}}/`a]) & \square & [] \\
  \tocdc\    & \psc \square \tr{\mult{N}}\ \tr{M}[\square \tr{\mult{N}}/`a] & \square  & \gp \\
  \tocdc\    & \tr{M}[\square \tr{\mult{N}}/`a] & \square & \square \tr{\mult{N}}:\gp \\
  \tocdc\    & \tr{M}[\square \tr{\mult{N}}/`a] & \square \tr{\mult{N}} & \gp \\
\end{array}
\]
This final state is $P$. Now we must prove $\tr{N} \to^{*} P$:
\[
\begin{array}{lcrr}
             & \tr{`m`a.[`a]M\{[`a]M'\mult{N}/[`a]M'\}\mult{N}} \\
  \triangleq & \wsc \gp `l`a.\pp \gp (\psc `a\ \tr{M[[`a]M'\mult{N}/[`a]M']\mult{N}}) & \square & \gp \\
  \tocdc\    & `l`a.\pp \gp (\psc `a\ \tr{M[[`a]M'\mult{N}/[`a]M']\mult{N}})(\square) & \square & [] \\
  \tocdc\    & (\pp \gp (\psc `a\ \tr{M[[`a]M'\mult{N}/[`a]M']\mult{N}}))[\square/`a] & \square & [] \\
  \tocdc\    & \pp \gp (\psc `a\ \tr{M[[`a]M'\mult{N}/[`a]M']\mult{N}})[\square/`a]   & \square & [] \\
  \tocdc\    & (\psc `a\ \tr{M[[`a]M'\mult{N}/[`a]M']\mult{N}})[\square/`a]           & \square & \gp \\
  \tocdc\    & \psc \square\ \tr{M[[`a]M'\mult{N}/[`a]M']\mult{N}}[\square/`a]        & \square & \gp \\
  \tocdc\    & \tr{M[[`a]M'\mult{N}/[`a]M']\mult{N}}[\square/`a]  & \square           & \gp \\
  \tocdc\    & \tr{M[[`a]M'\mult{N}/[`a]M']}[\square/`a]\ \tr{\mult{N}}               & \square & \gp \\
  \tocdc\    & \tr{M[[`a]M'\mult{N}/[`a]M']}[\square/`a]                              & \square \tr{\mult{N}} & \gp \\
  \tocdc\    & \tr{M}[\square \mult{N}/`a][\square/`a]                                & \square \tr{\mult{N}} & \gp \\
  \tocdc\    & \tr{M}[\square \mult{N}/`a]                                            & \square \tr{\mult{N}} & \gp \\
\end{array}
\]
In this final reduction step, $`a$ can it occur in $\tr{M}[\square \mult{N}/`a]$ because all occurrences have been substituted for $\square \tr{\mult{N}}$.

\[
\begin{array}{lcrr}
  \prooflabel{(`m`a.[`b]M)\mult{N} \to `m`a.[`b](M\{[`a]M'N/[`a]M'\})} & (`a \not= `b) \\
             & \tr{(`m`a.[`b]M)\mult{N}} \\
  \triangleq & \tr{(`m`a.[`b]M)}\tr{\mult{N}} \\
  \triangleq & (\wsc \gp `l`a.\pp \gp (\psc `b\  \tr{M})) \tr{\mult{N}} & \square & \gp \\
  \tocdc\    & (\wsc \gp `l`a.\pp \gp (\psc `b\  \tr{M})) & \square \tr{\mult{N}} & \gp \\
  \tocdc\    & (`l`a.\pp \gp (\psc `b\ \tr{M}))(\square \tr{\mult{N}}) & \square & []   \\
  \tocdc\    & (\pp \gp (\psc `b\ \tr{M}))[\square \tr{\mult{N}}/`a]   & \square & []   \\
  \tocdc\    & \pp \gp (\psc `b\ \tr{M})[\square \tr{\mult{N}}/`a]     & \square & []   \\
  \tocdc\    & (\psc `b\ \tr{M})[\square \tr{\mult{N}}/`a]             & \square & \gp  \\
  \tocdc\    & \psc `b\ \tr{M}[\square \tr{\mult{N}}/`a]               & \square & \gp  \\
  \tocdc\    & \tr{M}[\square \tr{\mult{N}}/`a]                        & \square & `b:\gp  \\
\end{array}
\]
\[
\begin{array}{lcrr}
             & \tr{`m`a.[`b](M\{[`a]M'N/[`a]M'\})} \\
  \triangleq & \wsc \gp `l`a. \pp \gp (\psc `b \tr{M\{[`a]M'N/[`a]M'\}})   & \square & \gp \\
  \tocdc\    & (`l`a. \pp \gp (\psc `b \tr{M\{[`a]M'N/[`a]M'\}}))(\square) & \square & [] \\
  \tocdc\    & (\pp \gp (\psc `b \tr{M\{[`a]M'N/[`a]M'\}}))[\square/`a]    & \square & [] \\
  \tocdc\    & \pp \gp (\psc `b \tr{M\{[`a]M'N/[`a]M'\}})[\square/`a]      & \square & [] \\
  \tocdc\    & \psc `b \tr{M\{[`a]M'N/[`a]M'\}}[\square/`a]                & \square & \gp \\
  \tocdc\    & \tr{M\{[`a]M'N/[`a]M'\}}[\square/`a]                        & \square & `b:\gp \\
  \tocdc\    & M[\square \tr{N}/`a][\square/`a]                          & \square & `b:\gp \\
  \tocdc\    & M[\square \tr{N}/`a]                                      & \square & `b:\gp \\
\end{array}
\]
\[
\begin{array}{lcrr}
  \prooflabel{`m`a.[`b]`m`g.[`d]M \to `m`a[`d]M[`b/`g]} & (`g \not= `d) \\
             & \tr{`m`a.[`b]`m`g.[`d]M} \\
  \triangleq & \wsc \gp `l`a.\pp \gp (\psc `b\ \tr{`m`g.[`d]M})     & \square & \gp        \\
  \tocdc\    & (`l`a.\pp \gp (\psc `b\ \tr{`m`g.[`d]M}))(\square)   & \square & []         \\
  \tocdc\    & (\pp \gp (\psc `b\ \tr{`m`g.[`d]M}))[\square/`a]     & \square & []         \\
  \tocdc\    & \pp \gp (\psc `b\ \tr{`m`g.[`d]M})[\square/`a]       & \square & []         \\
  \tocdc\    & (\psc `b\ \tr{`m`g.[`d]M})[\square/`a]               & \square & \gp        \\
  \tocdc\    & \psc `b\ \tr{`m`g.[`d]M}[\square/`a]                 & \square & \gp        \\
  \tocdc\    & \tr{`m`g.[`d]M}[\square/`a]                          & \square & `b:\gp  \\
  \triangleq & (\wsc \gp `l`g.\pp \gp (\psc `d \tr{M}))[\square/`a] & \square & `b:\gp  \\
  \tocdc\    & \wsc \gp `l`g.\pp \gp (\psc `d \tr{M})[\square/`a]   & \square & `b:\gp  \\
  \tocdc\    & (`l`g.\pp \gp (\psc `d \tr{M})[\square/`a])(`b)      & \square & []  \\
  \tocdc\    & (\pp \gp (\psc `d \tr{M})[\square/`a])[`b/`g]        & \square & []  \\
  \tocdc\    & \pp \gp (\psc `d \tr{M})[\square/`a][`b/`g]          & \square & []  \\
  \tocdc\    & (\psc `d \tr{M})[\square/`a][`b/`g]                  & \square & \gp  \\
  \tocdc\    & \psc `d \tr{M}[\square/`a][`b/`g]                    & \square & \gp  \\
  \tocdc\    & \tr{M}[\square/`a][`b/`g]                            & \square & `d:\gp  \\
\\
             & \tr{`m`a[`d]M[`b/`g]} \\
  \triangleq & \wsc \gp `l`a.\pp \gp (\psc `d\ \tr{M}[`b/`g])   & \square & \gp       \\
  \tocdc\     & (`l`a.\pp \gp (\psc `d\ \tr{M}[`b/`g]))(\square) & \square & []        \\
  \tocdc\     & (\pp \gp (\psc `d\ \tr{M}[`b/`g]))[\square/`a]   & \square & []        \\
  \tocdc\     & \pp \gp (\psc `d\ \tr{M}[`b/`g])[\square/`a]     & \square & []        \\
  \tocdc\     & (\psc `d\ \tr{M}[`b/`g])[\square/`a]             & \square & \gp       \\
  \tocdc\     & \psc `d\ \tr{M}[`b/`g][\square/`a]               & \square & \gp       \\
  \tocdc\     & \tr{M}[`b/`g][\square/`a]                        & \square & `d:\gp \\
\end{array}
\]
\begin{remark}
  The terms $\tr{M}[`b/`g][\square/`a]$ and $\tr{M}[\square/`a][`b/`g]$ are identical if $`a \not= `b \not= `g$.
\end{remark}

\[
\begin{array}{lcrr}
  \prooflabel{`m`a.[`b]`m`g.[`g]M \to `m`a[`b]M[`b/`g]} \\
             & \tr{`m`a.[`b]`m`g.[`g]M} \\
  \triangleq & \wsc \gp `l`a.\pp \gp (\psc `b\ \tr{`m`g.[`g]M})     & \square & \gp        \\
  \tocdc\    & (`l`a.\pp \gp (\psc `b\ \tr{`m`g.[`g]M}))(\square)   & \square & []         \\
  \tocdc\    & (\pp \gp (\psc `b\ \tr{`m`g.[`g]M}))[\square/`a]     & \square & []         \\
  \tocdc\    & \pp \gp (\psc `b\ \tr{`m`g.[`g]M})[\square/`a]       & \square & []         \\
  \tocdc\    & (\psc `b\ \tr{`m`g.[`g]M})[\square/`a]               & \square & \gp        \\
  \tocdc\    & \psc `b\ \tr{`m`g.[`g]M}[\square/`a]                 & \square & \gp        \\
  \tocdc\    & \tr{`m`g.[`g]M}[\square/`a]                          & \square & `b:\gp  \\
  \triangleq & (\wsc \gp `l`g.\pp \gp (\psc `g \tr{M}))[\square/`a] & \square & `b:\gp  \\
  \tocdc\    & \wsc \gp `l`g.\pp \gp (\psc `g \tr{M})[\square/`a]   & \square & `b:\gp  \\
  \tocdc\    & (`l`g.\pp \gp (\psc `g \tr{M})[\square/`a])(`b)      & \square & []  \\
  \tocdc\    & (\pp \gp (\psc `g \tr{M})[\square/`a])[`b/`g]        & \square & []  \\
  \tocdc\    & \pp \gp (\psc `g \tr{M})[\square/`a][`b/`g]          & \square & []  \\
  \tocdc\    & (\psc `g \tr{M})[\square/`a][`b/`g]                  & \square & \gp  \\
  \tocdc\    & \psc `b \tr{M}[\square/`a][`b/`g]                    & \square & \gp  \\
  \tocdc\    & \tr{M}[\square/`a][`b/`g]                            & \square & `b:\gp  \\
  \\
             & \tr{`m`a[`b]M[`b/`g]} \\
  \triangleq & \wsc \gp `l`a.\pp \gp (\psc `b\ \tr{M}[`b/`g])   & \square & \gp       \\
  \tocdc\     & (`l`a.\pp \gp (\psc `b\ \tr{M}[`b/`g]))(\square) & \square & []        \\
  \tocdc\     & (\pp \gp (\psc `b\ \tr{M}[`b/`g]))[\square/`a]   & \square & []        \\
  \tocdc\     & \pp \gp (\psc `b\ \tr{M}[`b/`g])[\square/`a]     & \square & []        \\
  \tocdc\     & (\psc `b\ \tr{M}[`b/`g])[\square/`a]             & \square & \gp       \\
  \tocdc\     & \psc `b\ \tr{M}[`b/`g][\square/`a]               & \square & \gp       \\
  \tocdc\     & \tr{M}[`b/`g][\square/`a]                        & \square & `b:\gp \\
\end{array}
\]
\end{Proof}

\begin{theorem}[Completeness of $\tr{`.}$]
\[
  \tr{M} \to^{nf} Q \Rightarrow \exists N. M \to^{nf} N \land \tr{N} \to^{nf} Q
\]
\end{theorem}
\begin{Proof}
By induction on the definition of $\tr{M}$.
\prooftitle{$`m`a.[`a]M$}
From the proof for Theorem \ref{theorem:soundness}, we know that:
\[
\begin{array}{rclll}
\tr{`m`a.[`a]M} & \to^{nf} & \tr{M} & \square & \gp
\end{array}
\]
From the definition of $\to_{`m}$:
\[
`m`a.[`a]M \to^{nf} M
\]
And from the translation in \ref{sec:lmu-in-cdc}:
\[
\begin{array}{rclll}
\tr{M} \triangleq \tr{M} & \square & \gp 
\end{array}
\]

\prooftitle{$(`m`a.[`a]M)N$}
\dots
\prooftitle{$(`m`a.[`b]M)N$ \hspace{1cm} ($`a \not= `b$)}
\dots
\prooftitle{$(`m`a.[`b]`m`g.[`d]M)$ \hspace{1cm} ($`g \not= `d$)}
\dots
\prooftitle{$(`m`a.[`b]`m`g.[`g]M)$}
\dots
\end{Proof}


\section{Interpreting \ltry\ in CDC}

  By appending the interpretation of \ltry\ in $`l`m$ with the interpretation of
  $`l`m$ in CDC, we get a translation from \ltry\ to CDC:

  % TODO: add lines over (catch n(x) = L)
  \definition{
    \textsc{Translation of \ltry\ into CDC}
    {
    \relscale{0.9}
    \[
    \begin{array}{rcl}
      \tr{x} & \triangleq & x \\
      \tr{`lx.M} & \triangleq & `lx.\tr{M} \\
      \tr{MN} & \triangleq & \tr{M} \tr{N} \\
      
      \tr{\throw{n($M$)}} & \triangleq & \wsc\ \gp\ `l\nonocc.\pp\ \gp\ (\psc\ n\ (c_n\ \tr{M})) \\ 
      \tr{\try M;\ \catch{n($x$) = $L$}}
        &
        \triangleq 
        &
        (`lc_n.\wsc\ \gp\ `ln.\pp\ \gp\ (\psc\ n\ \tr{M}))(`lx.\tr{L})
      \\ 
      %\multicolumn{3}{l}{\tr{\try M;\ \catch{n$_i$($x$) = $M_i$};\ \catch{m($x$) = $L$}}} \\
      \multicolumn{3}{l}{\tr{\try M;\ \mcatch;\ \catch{m($x$) = $L$}}} \\
        & \triangleq \\
      \multicolumn{3}{r}{
        %(`lc.\wsc\ \gp\ `lm.\pp\ \gp\ (\psc\ m\ \tr{\try M;\ \catch{n$_i$($x$) = $M_i$}}))(`lx.\tr{L})
        (`lc_m.\wsc\ \gp\ `lm.\pp\ \gp\ (\psc\ m\ \tr{\try M;\ \mcatch}))(`lx.\tr{L})
      }
    \end{array} 
    \]
    }
  }
