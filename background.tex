\chapter{Background}

\section{Formal Systems}

\section{\lam-Calculus}

\section{Haskell}

\section{Logic, Types, and their Computation Interpretation}
\subsection{Continuations}
\subsection{Delimited-Continuations}

\section{\lmu-Calculus}

\section{\ltry-Calculus}

\section{Delimited-Continuation Calculus}

Simon Peyton-Jones \textit{et al.}\ extended the \lam-calculus with additional operators in order create a framework for implementing delimited continuations \cite{JonesDS07}. This calculus will be referred to as the delimited-continuation calculus or DCC. Many calculi have been devised with control mechanisms. Like the \lmu-calculus, these control mechanisms are all specific instances of delimited and undelimited continuations. DCC provides a set of operations that are capable of expressing many of these other common control mechanisms.

The grammar of DCC is an extension of the standard \lam-calculus:

\begin{figure}[!h]
\begin{definition}[Grammar rules for DCC]
\[
\begin{array}{lrcl}
\textrm{(Variables)} & x, y, \dots \\
\textrm{(Expressions)} & e & ::= & x\ |\ `lx.e\ |\ e\ e^\prime \\
                       &   &  |  &  newPrompt\ |\ pushPrompt\ e\ e \\
                       &   &  |  &  withSubCont\ e\ e\ |\ pushSubCont\ e\ e
\end{array}
\]
\end{definition}
\end{figure}

% TODO: ensure prompts and continuation stack has been explained before reaching this point
The additional terms behave as follows:
\begin{itemize}
\item \op{newPrompt} returns a new and distinct prompt.
\item \op{pushPrompt}'s first argument is a prompt which is pushed onto the continuation stack before evaluating its second argument. 
\item \op{withSubCont} captures the subcontinuation from the most recent occurrence of the first argument (a prompt) on the excution stack to the current point of execution. Aborts this continuation and applies the second argument (a \lam-abstraction) to the captured continuation.
\item \op{pushSubCont} pushes the current continuation and then its first argument (a subcontinuation) onto the continuation stack before evaluating its second argument.
\end{itemize}
