\documentclass[11pt,a4paper,twoside,draft]{report}

\usepackage{titlesec}
\usepackage{graphicx}
\usepackage{amsthm}
\usepackage{qsymbols}
\usepackage{listings}
\usepackage{amsmath}
\usepackage{amssymb}
\usepackage{derivation}
\usepackage{theorems}
\usepackage{relsize}
\usepackage{fancyvrb}
\usepackage{verbatim}

\usepackage{macros}

\newcommand{\reporttitle}{Exception Handling in Haskell}
\newcommand{\name}{William S. Fisher}
\newcommand{\supervisor}{Steffen van Bakel}
\newcommand{\course}{Computing Science}

\definecolor{pale-gray}{gray}{0.75}
\newcommand{\hsp}{\hspace{15pt}}
\titleformat{\chapter}[hang]{\Huge\bfseries}{\thechapter\hsp\color{pale-gray}{|}\hsp}{0pt}{\Huge\bfseries}

\begin{document}

%\begin{titlepage}

\center

\textsc{\Large{Imperial College London}}\\[0.5cm] 
\textsc{\large{Department of Computing}}\\[0.5cm] 

\vspace{4cm}
\huge{\textbf{\reporttitle}}
\vspace{2cm}

\large{
  \emph{by} \\
  \vspace{0.1cm}
  \textbf{\name}
}

\vspace{1cm}

\large{
  \emph{supervised by} \\
  \vspace{0.1cm}
  \textbf{\supervisor}
}

\vfill % Fill the rest of the page with whitespace
\large{
Submitted in partial fulfillment of the requirements for the MSc degree in
\course~of Imperial College London\\[0.5cm]
}

\makeatletter
\@date 
\makeatother

\end{titlepage}

%\begin{abstract}
%Implementing exception handling in Haskell. Unlike other libraries, use named exception handlers. Use the \ltry-calculus to formalize and explore a series of translations between multiple calculi to arrive at a translation into Haskell. Explore properties of this translation including soundness and completeness. Publish useable Haskell library.
%\end{abstract}
%
%\pagenumbering{gobble}
%
%\chapter*{Acknowledgements}





%
%\tableofcontents
%\clearpage{\pagestyle{empty}\cleardoublepage}
%
%\pagenumbering{arabic}
%\setcounter{page}{1}

%\chapter{Introduction}

In 1936 Alonzo Church presented the \lam-calculus, a formal system for expressing the entire set of computable problems \cite{Church36}.\footnote{
  The ability of a system to express the entire set of computable problems is now commonly known as `Turing completeness' although here this description would be anachronistic; 
  Turing's work on computability was published a year after Church's.
}
Later, Haskell Curry and William Howard both discover a correspondence between the simply-typed \lam-calculus and intuitionistic logic. 
However the correspondence runs deeper than this: there is actually a true isomorphism between many systems of formal logic and computational calculi. 

The correspondence between systems of formal logic and computer programs drove research into both the logical counterparts of programs and the computational counterparts of logical systems.
As a part of this, Michel Parigot developed the \lmu-calculus, an extension of the \lam-calculus, 
through an isomorphism to classical logic \cite{Parigot92}.
More recently, Steffen van Bakel modelled exceptions and exception handling through an extension of the \lam-calculus called the \ltry-calculus \cite{Bakel15}.\footnote{
  Exceptions are a form of error generated by computers to indicate that the flow of control cannot continue as expected and must be aborted. 
}

Exceptions are most commonly found in two kinds: 
  systems where there is a single type of exception such as in Javascript, 
  and systems where exceptions are discriminated by \emph{type} as in Java and Haskell. 
Unlike either of these approaches, 
the exception-handling proposed by the \ltry-calculus introduces exceptions discriminated by name.

\subsubsection{Project}
This paper proposes an implementation of named-exceptions in Haskell.
First we define a translation from the \lmu-calculus to a calculus for expressing delimited continuations, CDC \cite{JonesDS07}
\[
\begin{array}{lcr}
  `l`m & \to & CDC
\end{array}
\]
CDC is a calculus which closely models Haskell's syntax and which has been extended with operations for manipulating delimited continuations.
The CDC library for Haskell defined by Jones \emph{et al.} in \cite{JonesDS07} evaluates CDC terms in a single step, returning the final value of the reduction.
However, we are interested in producing the entire step-by-step evaluation of CDC terms which we can transliterate into CDC derivations for verifying translations of \lmu\ into CDC.
This motivates us to write a term-rewriting program for evaluating CDC terms.
The proofs generated by this program are used to investigate the soundness and completeness of the $`l`m \to CDC$ translation.
The translation is then concatenated with van Bakel's original translation of his \ltry-calculus into the \lmu-calculus which yields a translation from \ltry\ to CDC:
\[
   (`l^{try} \to `l`m) \circ (`l`m \to CDC) = `l^{try} \to CDC
\]
We use this final translation to explore implementations of named exceptions directly in Haskell.

\subsubsection{Outline}
Chapter \ref{chapter:background} discusses the related context for this project. Following this, this paper makes the following original contributions:
\begin{itemize}
\item A Haskell interpreter for the calculus of delimited continuations, CDC (in Chapter \ref{chapter:interpreter})
\item A translation of \lmu\ into CDC along with proofs of its soundness and completeness with respect to $`m$-reduction (in Chapter \ref{chapter:translations})
\item A translation of \ltry\ into CDC (also in Chapter \ref{chapter:translations}) 
\item A proof of concept implementation of \ltry\ in Haskell, based on this translation (in Chapter \ref{chapter:implementation})
\item The specification for a language extension for Haskell with named exceptions, based on \ltry\ (also Chapter \ref{chapter:implementation})
\end{itemize}

%include{background}
%\chapter{CDC Interpreter}

This chapter explores the implementation of an interpreter for CDC. Portions of source code are examined in detail although the full source can be found in the appendix.

\section{Implementation}

Although Peyton-Jones \emph{et al.} implement a language-level module for CDC, 
we are interested in the intermediate term transformations.
Using the step-by-step transformations produced by this interpreter,
we can construct and verify the implementations of \ltry\ and \lmu\ into CDC.
Examining transformation steps in full also allows us to derive proofs of soundness and completeness for these translations. 
For this reason, the interpreter was implemented as a term-rewriting program.

We follow the operational semantics of the system to provide an implementation.
This is not necessary and results in an ineffecient implementation.
Despite this, it is the simplest approach to implementation and efficiency is not central to producing proofs.

\subsection{Data structures}
There are two data types for representing CDC terms, \mono{Value} and \mono{Expr}
(Figure \ref{fig:cdc-data-structures}).
Values are not evaluated: when a term has been reduced to a value, 
it has terminated on that value. 
An expression (\mono{Expr}) is a term that can be evaluated to another term. 
The only exception is a \mono{Hole} which can take any position an expression can. 
For this reason, it must be a data constructor for expression types.

The core of the abstract machine is a function from one state to the next. 
A state is its own data type which corresponds to the tuple from the specification of the semantics of the abstract machine $\langle e,\ D,\ E,\ q\rangle$.

\begin{figure}\label{fig:cdc-data-structures}
\Verbatimcode
data Value = Var Char
  | Abs Char Expr
  | Prompt Int
  | Seq [Expr]
  deriving (Show, Eq)
  
data Expr = Val Value 
  | App Expr Expr
  | Hole
  | PushPrompt Expr Expr
  | PushSubCont Expr Expr
  | WithSubCont Expr Expr
  | NewPrompt
  
  | Sub Expr Expr Char
  deriving (Show, Eq)

data State = State Expr Expr [Expr] Value
  deriving (Show, Eq)
\end{Verbatim}
\caption{Data structures for the CDC interpreter}
\end{figure}

\subsection{Utility Functions}
Some utility functions are simplify the implementation.
Informally, these functions behave as follows (see Figure ~\ref{fig:cdc-helper-functions} for implementations details):

\begin{itemize}\itemsep0.1cm

\item \mono{prettify :: Expr -> String} is defined inductively for pretty-\\ printing terms.

\item \mono{ret :: Expr -> Expr -> Expr} returns the first expression with any holes filled in by the second expression.

\item \mono{composeContexts :: [Expr] -> Expr} takes a sequence of expressions and, starting from the end, 
fills the hole of each expression with the previous expression. 
This in effect joins the output of each context with the input of the next context.

\item \mono{promptMatch :: Int -> Expr -> Bool} returns true if the second argument is a Prompt with the same value as the first argument

\item \mono{splitBefore :: [Expr] -> Int -> [Expr]} returns the sequence of expressions up until (but not including) the prompt matching the second argument.

\item \mono{splitAfter :: [Expr] -> Int -> [Expr]} returns the sequence of expressions from (but not including) the prompt matching the second argument.

\item \mono{sub :: Expr -> Expr -> Char -> Expr} returns the first expression with all occurences of the third expression replaced by the second expression. 
If we name the arguments \mono{sub M V x} then this corresponds to the result of evaluating the substitution notation $M[v/x]$.
\end{itemize}

\begin{figure}\label{fig:cdc-helper-functions}
\Verbatimcode
  ret :: Expr -> Expr -> Expr
  ret d e = case d of
    Hole -> e
    App m n -> App (ret m e) (ret n e)
    Val (Abs x m) -> Val $ Abs x (ret m e)
    PushPrompt m n -> PushPrompt (ret m e) (ret n e)
    WithSubCont m n -> WithSubCont (ret m e) (ret n e)
    PushSubCont m n -> PushSubCont (ret m e) (ret n e)
    otherwise -> d

  composeContexts :: [Expr] -> Expr
  composeContexts = foldr ret Hole . reverse

  sub :: Expr -> Expr -> Char -> Expr
  sub m v x = case m of
    Val (Var n) -> if n == x then v else m
    Val (Abs y e) -> Val (Abs y $ sub e v x)
    Val (Prompt p) -> Val (Prompt p)
    App e e' -> App (sub e v x) (sub e' v x)
    NewPrompt -> NewPrompt
    PushPrompt e e' -> PushPrompt (sub e v x) (sub e' v x)
    WithSubCont e e' -> WithSubCont (sub e v x) (sub e' v x)
    PushSubCont e e' -> PushSubCont (sub e v x) (sub e' v x)

  promptMatch :: Int -> Expr -> Bool
  promptMatch i p = case p of
    (Val (Prompt p')) -> i == p'
    otherwise -> False
 
  splitBefore :: [Expr] -> [Expr]
  splitBefore p es = takeWhile (not . promptMatch p) es

  splitAfter :: [Expr] -> [Expr]
  splitAfter  p es = case length es of
    0 -> []
    otherwise -> tail list
    where list = dropWhile (not . promptMatch p) es
\end{Verbatim}
\caption{Utility functions for CDC interpreter}
\end{figure}

% TODO: make this section a little more structured
\subsection{Reduction Rules}

The heavy lifting of the interpreter is done by the function \mono{eval :: State -> State}. 
\mono{eval} is defined inductively on the structure of CDC terms. 
Using pattern-matching, each case of \mono{eval} corresponds directly to at least one of the reduction rules of the CDC abstract machine. 

\subsubsection{Application}
{
\relscale{0.9}
\[
\begin{array}{lcll}
\langle e\ e^\prime, D, E, q \rangle &\to &\langle e, D[\square\ e^\prime], E, q \rangle &\text{e non-value} \\
\langle v\ e, D, E, q \rangle &\to &\langle e, D[v\ \square], E, q \rangle &\text{e non-value} \\
\langle (`lx.e)\ v, D, E, q \rangle &\to &\langle e[v/x], D, E, q \rangle \\
\end{array}
\]
}
The \mono{App e e'} case deals with applications:
if both terms are values and the first term is an abstraction of the form \mono{$`l$x.m}, 
the dominant term becomes a substitution of \mono{e'} for \mono{x} in \mono{m}. 
Otherwise, the term that is not a value is made the dominant term and the remainder of the application is added to the current context. 
If both terms are redexes, the left-most is made the dominant term first. 
In effect, an application first ensures the left-hand term has been evaluated fully before evaluating the right-hand term.
\Verbatimcode
eval (State (App e e') d es q) = case e of
  Val v -> case e' of 
    Val _ -> case v of 
      Abs x m -> State (Sub m e' x) d es q
      Seq es' -> State (ret (composeContexts es') e') d es q
      otherwise -> State (App e e') d es q
    otherwise -> State e' (ret d (App e Hole)) es q
  otherwise -> State e (ret d (App Hole e')) es q
\end{Verbatim}

\subsubsection{PushPrompt}
{
\relscale{0.9}
\[
\begin{array}{lcll}
\langle pushPrompt\ e\ e^\prime, D, E, q \rangle &\to &\langle e, D[pushPrompt\ \square\ e^\prime], E, q \rangle &\text{e non-value} \\
\langle pushPrompt\ p\ e, D, E, q \rangle &\to &\langle e, \square, p : D : E, q \rangle \\
\end{array}
\]
}

The \mono{PushPrompt e e'} case ensures the left term is a value.
It then pushes the first argument (a prompt) and the current context onto the stack and makes the second argument the dominant term. 
\Verbatimcode
eval (State (PushPrompt e e') d es q) = case e of
  Val _ -> State e' Hole (e:d:es) q
  otherwise -> case d of
    Hole -> State e (PushPrompt Hole e') es q
    otherwise -> State e (ret d (PushPrompt Hole e')) es q
\end{Verbatim}

\subsubsection{WithSubCont}
{
\relscale{0.9}
\[
\begin{array}{lcll}
\langle withSubCont\ e\ e^\prime, D, E, q \rangle &\to &\langle e, D[withSubCont\ \square\ e^\prime], E, q \rangle &\text{e non-value} \\
\langle withSubCont\ p\ e, D, E, q \rangle &\to &\langle e, D[withSubCont\ p\ \square], E, q \rangle &\text{e non-value} \\
\langle withSubCont \ p\ v, D, E, q \rangle &\to &\langle v (D : E\until{p}, \square, E\from{p}, q \rangle \\
\end{array}
\]
}

The reduction rules for \mono{WithSubCont e e'} ensure that the first argument has been evaluated to a prompt \mono{p} and then that the second argument has been evaluated to an abstraction. 
Finally, it appends the current continuation to the sequence yielded by splitting the continuation stack at \mono{p}, 
and creates an application of the second argument to this sequence.
\Verbatimcode
eval (State (WithSubCont e e') d es q) = 
  case e of
    Val v -> case e' of
      Val _ -> case v of (Prompt p) -> 
        State (App e' (seq' (d:beforeP))) Hole afterP q
          where beforeP = splitBefore p es
                afterP = splitAfter p es
      otherwise -> State e' (ret d (WithSubCont e Hole)) es q 
    otherwise -> State e (ret d (WithSubCont Hole e')) es q 
\end{Verbatim}

\subsubsection{PushSubCont}
{
\relscale{0.9}
\[
\begin{array}{lcll}
\langle pushSubCont\ e\ e^\prime, D, E, q \rangle &\to &\langle e, D[pushSubCont\ \square\ e^\prime], E, q \rangle &\text{e non-value} \\
\langle pushSubCont E^\prime\ e, D, E, q \rangle &\to &\langle e, \square, E^\prime \app (D : E), q \rangle \\
\end{array}
\]
}
Reducing \mono{PushSubCont e e'} ensures that the first argument is a sequence.
Then it pushes the current continuation, followed by this sequence, onto the stack.
The second argument is promoted to be the dominant term.
This has the effect of evaluating the dominant term and return the result to the sequence.
\Verbatimcode
eval (State (PushSubCont e e') d es q) = 
  case e of
    Val (Seq es') -> State e' Hole (es'++(d:es)) q
    otherwise -> State e (ret d (PushSubCont Hole e')) es q
\end{Verbatim}

\subsubsection{Substitution}
The reduction of \mono{Sub e y x} uses \mono{sub} to recursively substitute the third argument with the second in the first. 
\Verbatimcode
eval (State (Sub e y x) d es q) = 
  State (sub e y x) d es q
\end{Verbatim}

\subsubsection{NewPrompt}
{
\relscale{0.9}
\[
\begin{array}{lcll}
\langle newPrompt, D, E, q \rangle &\to &\langle q, D, E, q+1 \rangle
\end{array}
\]
}
Evaluating \mono{NewPrompt} places the value of the current prompt as the dominant term and increments the global prompt counter:
\Verbatimcode
eval (State NewPrompt d es (Prompt p)) = 
  State (Val (Prompt p)) d es (Prompt $ p+1)
\end{Verbatim}

%\Verbatimcode
eval :: State -> State
eval (State (App e e') d es q) = case e of
  Val v -> case e' of 
    Val _ -> case v of 
      Abs x m -> State (Sub m e' x) d es q
      Seq es' -> State (ret (composeContexts es') e') d es q
      otherwise -> State (App e e') d es q
    otherwise -> State e' (ret d (App e Hole)) es q
  otherwise -> State e (ret d (App Hole e')) es q

eval (State (PushPrompt e e') d es q) = case e of
  Val _ -> State e' Hole (e:d:es) q
  otherwise -> case d of
    Hole -> State e (PushPrompt Hole e') es q
    otherwise -> State e (ret d (PushPrompt Hole e')) es q

eval (State (WithSubCont e e') d es q) = case e of
  Val v -> case e' of
    Val _ -> case v of (Prompt p) -> State (App e' (seq' (d:beforeP))) 
                                            Hole afterP q
                                     where beforeP = splitBefore p es
                                           afterP = splitAfter p es
    otherwise -> State e' (ret d (WithSubCont e Hole)) es q 
  otherwise -> State e (ret d (WithSubCont Hole e')) es q 
  
eval (State (PushSubCont e e') d es q) = case e of
  Val (Seq es') -> State e' Hole (es'++(d:es)) q
  otherwise -> State e (ret d (PushSubCont Hole e')) es q

eval (State (Sub e y x) d es q) = State e' d es q
  where e' = case e of
          Val (Var m) -> if m == x then y else (Val (Var m))
          Val (Abs h m) -> Val (Abs h (sub m y x))
          App m n -> App (sub m y x) (sub n y x)
          Val (Prompt p) -> Val (Prompt p)
          NewPrompt -> NewPrompt
          PushPrompt e1 e2 -> PushPrompt (sub e1 y x) (sub e2 y x)
          WithSubCont e1 e2 -> WithSubCont (sub e1 y x) (sub e2 y x)
          PushSubCont e1 e2 -> PushSubCont (sub e1 y x) (sub e2 y x)

eval (State NewPrompt d es (Prompt p)) = State (Val (Prompt p)) 
                                               d es (Prompt $ p+1)

eval (State (Val v) d es q) = case d of
  Hole -> case es of
    (e:es') -> case e of
      (Val (Prompt p)) -> State (Val v) Hole es' q
      otherwise -> State (Val v) e es' q
    otherwise -> State (Val v) d es q
  otherwise -> State (ret d (Val v)) Hole es q

\end{Verbatim}


%TODO: show example output for a given term
%TODO: evaluation of different implementation approaches

\chapter{Translations}

In this chapter, we develop a interpretation of \lmu\ in DCC.
We prove some properties of this interpretation, including \emph{soundness}.
We concatenate this interpretation with van Bakel's interpretation of \ltry\ in \lmu.
This concatenation yields an interpretation of \ltry\ in DCC.
This will then be used as a basis for the implementation of \ltry\ in Haskell.

\section{Interpreting \ltry\ in \lmu}

Steffen van Bakel describes the interpretation of \ltry\ to \lmu:

\[
  \begin{array}{rcl}
    \tr{x} &\triangleq& x \\
    \tr{`lx.M} &\triangleq& `lx.\tr{M} \\
    \tr{M N} &\triangleq& \tr{M}\tr{N} \\
    \multicolumn{3}{l}{\tr{\try M;\ \mcatch;\ \catch{m($x$) = $L$}}} \\
    & \triangleq & \\
    &\multicolumn{2}{l}{ (`lc_m.`m\text{m}.[\text{m}]\tr{\try M;\ \mcatch})(`lx.\tr{L})} \\
    
    \tr{\try M;\ \catch{m($x$) = $L$}} & \triangleq & (`lc_m.`m\text{m}.[\text{m}]\tr{M})(`lx.\tr{L}) \\
    \tr{\throw{n($M$)}} &\triangleq& `l\nonocc.[\text{n}]c_n\tr{M}
  \end{array}
\]

$\throw{n($M$)}$ terms are modelled using \lmu-abstractions of non-occurring names. This has the effect of removing all terms it is applied to:

\[
  (`m\nonocc.M)NOP \to (`m\nonocc.M)OP \to (`m\nonocc.M)P \to `m\nonocc.M
\]

The contents of the \lmu-abstraction calls $c_n$.
This \lam-variable is bound by the translation of \textbf{try} terms.
This binding means that the exception handlers, represented by $`lx.\tr{L}$,
are in scope for the reduction of the body of the try $M$.

% TODO: explain translation ?

\section{Interpreting \lmu\ in DCC}

% TODO: explain initial attempt and explain why it was simple and intuitive
%       and follow up with why it went wrong and what we had to change.
%       surprisingly did not work. replacing alpha in wsc terms to distribute
%       context throughout is why new one works

%\begin{figure}[!h]
%\definition{
%  \textsc{(Interpretation of \lmu\ into DCC)}
%  \[
%  \begin{array}{lcl}
%    \dbr{x}        & \triangleq & x \\
%    \dbr{`lx.M}    & \triangleq & `lx.\dbr{M} \\
%    \dbr{M N}      & \triangleq & \dbr{M} \dbr{N} \\
%    \dbr{`m`a.M}   & \triangleq & (`lp.\pp\ p\ \\
%    &&                 \hspace{1cm} ((\wsc\ p\ (`l`a.\dbr{M})) \\ 
%    &&                 \hspace{1cm} \dbr{N}) \\
%    &&               )\ \np \\
%    \dbr{[`b]M} & \triangleq & \psc\ `b\ \dbr{M}
%  \end{array}
%  \]
%}
%\end{figure}

The translation of \lmu-terms into DCC assumes that there is a single global prompt \gp. 
It also assumes that this prompt has already been pushed onto the stack.
This means that the translation of a full \lmu-program $M$ in DCC is:

\definition{
  \textsc{(Initialization of stack for running $M$ in DCC)}
  \[ (`l\gp.\pp\ \gp\ \dbr{M})\ \np \]
}
This creates a new prompt \gp\ which is in scope for all terms in $M$.
It also prepares the stack by pushing \gp\ immediately. 
With the stack prepared, 
the interpretation of \lmu\ terms into DCC proceeds as follows:

\definition{
  \textsc{(Interpretation of \lmu\ into DCC)}
  \[
  \begin{array}{lcl}
    \tr{x}        & \triangleq & x \\
    \tr{`lx.M}    & \triangleq & `lx.\tr{M} \\
    \tr{M N}      & \triangleq & \tr{M} \tr{N} \\
    \tr{`m`a.M}   & \triangleq & \wsc\ \gp\ `l`a.\pp\ \gp\ \tr{M} \\
    \tr{[`b]M} & \triangleq & \psc\ `b\ \tr{M}
  \end{array}
  \]
}

To implement \lmu-abstractions, we capture the subcontinuation until the last occurrence of \gp\ on the stack.
This subcontinuation is bound to $`a$ which ensures the subcontinuation is distributed to all occurrences of $`a$ in $M$.
\gp\ is then pushed back onto the stack before the evaluation of $M$.

To implement named-terms, the subcontinuation $`b$ is pushed into the stack before evaluating $M$.
This means the reduct of $M$ will be returned to this subcontinuation.
In effect, this reduces $M$ and passes the result to $`b$. 

\begin{example}{$\tr{`m`a.[`a](`lx.x)} \to \tr{`lx.x}$}
\[
\begin{array}{llrr}
             & \tr{`m`a.[`a](`lx.x)}  \\
  \triangleq & \wsc\ \gp\ `l`a.\pp\ \gp\ `lx.x, & \square, & \gp:[] \\
  \tob\      & (`l`a.\pp\ \gp\ `lx.x)(\square), & \square, & [] \\
  \tob\      & (\pp\ \gp\ `lx.x)[\square/`a],   & \square, & [] \\
  \tob\      & \pp\ \gp\ `lx.x,                 & \square, & [] \\
  \tob\      & `lx.x,                           & \square, & \gp:[]
\end{array}   
\]
The final state has restored the initial state of the stack by pushing \gp\ back on.
\end{example}


% TODO: Add proofs for completeness and other properties
\begin{theorem}[Soundness of $\tr{\bullet}$]
If $M \rightarrow_{`m} N$ then $\dbr{M} \rightarrow_{DCC} \dbr{N}$
\end{theorem}

% TODO: separate D and E in DCC proofs 
\begin{proof}{By induction on the definition of $\rightarrow_{`m}$}
\[
\begin{array}{rlrl}
  \prooflabel{(`lx.M)N \to M[N/x]:} \\
               & \tr{(`lx.M)N} \\
    \triangleq & \tr{(`lx.M)} \tr{N} \\
    \triangleq & (`lx.\tr{M}) \tr{N}, & \square, & \gp:[] \\
    \tob\   & \tr{M}[\tr{N}/x], & \square, & \gp:[] \\
    \triangleq\ & \tr{M[\tr{N}/x]}, & \square, & \gp:[] \\
\end{array} 
\]
\\
\[
\begin{array}{rlrr}
  \prooflabel{(`m`a.[`b]M)N \to `m`a.([`b]M)[[`a]M^\prime N/[`a]M^\prime]:} \\
             & \tr{(`m`a.[`b]M)N} \\
  \triangleq & \tr{(`m`a.[`b`]M)} \tr{N} \\
  \triangleq & (\wsc\ \gp\ `l`a. \pp\ \gp\ (\psc\ `b\ \tr{M})) \tr{N}, & \square, & \gp:[] \\
  \todcc\ & \wsc\ \gp\ `l`a. \pp\ \gp\ (\psc\ `b\ \tr{M}), & \square \tr{N}, & \gp:[] \\
  \todcc\ & (`l`a. \pp\ \gp\ (\psc\ `b\ \tr{M}))(\square\tr{N}), & \square, & [] \\
  \tob\ & (\pp\ \gp\ (\psc\ `b\ \tr{M}))[\square\tr{N}/`a], & \square, & [] \\
  \tob\ & \pp\ \gp\ (\psc\ `b\ (\tr{M}[\square\tr{N}/`a])), & \square, & [] \\
  \todcc\ & \psc\ `b\ (\tr{M}[\square\tr{N}/`a]), & \square, & \gp:[] \\
  \triangleq\ & \tr{`m`a.([`b]M)[[`a]M^\prime N/[`a]M^\prime]} \\
\end{array} 
\]
\\
\[
\begin{array}{rlrr}
  \prooflabel{`m`a.[`a]M \to M:} \\
              & \tr{`m`a.[`a]M} \\
   \triangleq & \wsc\ \gp\ `l`a.\pp\ \gp\ (\psc\ `a\ \tr{M}), & \square, & \gp:[]  \\
   \todcc\    & `l`a.\pp\ \gp\ (\psc\ `a\ \tr{M})(\square), & \square, & [] \\
   \tob\      & \pp\ \gp\ (\psc\ `a\ \tr{M})[\square/`a], & \square, & [] \\
   \tob\      & \pp\ \gp\ (\psc\ \square\ (\tr{M}[\square/`a]), & \square, & [] \\
   \todcc\    & \psc\ \square\ (\tr{M}[\square/`a], & \square, & \gp:[] \\
   \todcc\    & \tr{M}[\square/`a], & \square,  & \gp:[] \\
   \triangleq & \tr{M} \\
\end{array}
\]
\\ 
\[
\begin{array}{rlrr}
  \prooflabel{`m`a.[`b]`m`g.[`d]M \to `m`a.[`d](M[`b/`g]):} \\
    \triangleq & \wsc\ \gp\ `l`a.\pp\ \gp\ \tr{[`b]`m`g.[`d]M}, & \square, & \gp:[] \\
    \todcc\    & `l`a.\pp\ \gp\ \tr{[`b]`m`g.[`d]M}(\square), & \square, & [] \\
    \tob\      & (\pp\ \gp\ \tr{[`b]`m`g.[`d]M})[\square/`a], & \square,& [] \\
    \tob\      & \pp\ \gp\ \tr{[`b]`m`g.[`d]M}[\square/`a], & \square,& [] \\
    \tob\      & \tr{[`b]`m`g.[`d]M}[\square/`a], & \square,& \gp:[] \\
    \triangleq & (psc `b \tr{`m`g.[`d]M})[\square/`a], & \square,& \gp:[] \\
    \triangleq & \tr{`m`g.[`d]M})[\square/`a], & \square, & `b:\gp:[] \\
    \triangleq & (\wsc\ \gp\ `l`g.\tr{[`d]M})[\square/`a], & \square,& `b:\gp:[] \\
    \triangleq & \wsc\ \gp\ `l`g.\pp\ \gp\ \tr{[`d]M}[\square/`a], & \square, & `b:\gp:[] \\
    \triangleq & (`l`g.\pp\ \gp\ \tr{[`d]M}[\square/`a])(`b), & \square, & [] \\
    \triangleq & (\pp\ \gp\ \tr{[`d]M}[\square/`a])[`b/`g], & \square,& [] \\
    \triangleq & \pp\ \gp\ (\tr{[`d]M}[\square/`a])[`b/`g], & \square,& [] \\
    \triangleq & \tr{[`d]M}[\square/`a])[`b/`g], & \square,& \gp:[] \\
    \triangleq & (\psc\ `d\ M[\square/`a][`b/`g], & \square,& \gp:[] \\
    \triangleq & \psc\ `d\ (M[\square/`a])[`b/`g], & \square,& \gp:[] \\
    \triangleq & \tr{`m`a.[`d](M[`b/`g])}, & \square,& \gp:[] \\
\end{array}
\]
\\ 
\[
\begin{array}{rlrl}
  \prooflabel{`m`a.[`b]`m`g.[`g]M \to `m`a.[`b](M[`b/`g]):} \\
               & \tr{`m`a.[`b]`m`g.[`d]M} \\
    \triangleq & \wsc\ \gp\ `l`a.\pp\ \gp\ \tr{[`b]`m`g.[`d]M},    & \square, & \gp:[] \\
    \todcc\    & `l`a.\pp\ \gp\ \tr{[`b]`m`g.[`d]M}(\square),    & \square, & [] \\
    \tob\      & (\pp\ \gp\ \tr{[`b]`m`g.[`d]M})[\square/`a],    & \square, & [] \\
    \tob\      & \pp\ \gp\ \tr{[`b]`m`g.[`d]M}[\square/`a],      & \square, & [] \\
    \tob\      & \tr{[`b]`m`g.[`d]M}[\square/`a],              & \square, & \gp:[] \\
    \triangleq & (psc `b \tr{`m`g.[`d]M})[\square/`a],         & \square, & \gp:[] \\
    \triangleq & \tr{`m`g.[`d]M})[\square/`a],                 & \square, & `b:\gp:[] \\
    \triangleq & (\wsc\ \gp\ `l`g.\tr{[`d]M})[\square/`a],       & \square, & `b:\gp:[] \\
    \triangleq & \wsc\ \gp\ `l`g.\pp\ \gp\ \tr{[`d]M}[\square/`a], & \square, & `b:\gp:[] \\
    \triangleq & (`l`g.\pp\ \gp\ \tr{[`d]M}[\square/`a])(`b),    & \square, & [] \\
    \triangleq & (\pp\ \gp\ \tr{[`d]M}[\square/`a])[`b/`g],      & \square, & [] \\
    \triangleq & \pp\ \gp\ (\tr{[`d]M}[\square/`a])[`b/`g],      & \square, & [] \\
    \triangleq & \tr{[`d]M}[\square/`a])[`b/`g],               & \square, & \gp:[] \\
    \triangleq & (\psc\ `d\ M[\square/`a][`b/`g],              & \square, & \gp:[] \\
    \triangleq & \psc\ `b\ (M[\square/`a])[`b/`g],             & \square, & \gp:[] \\
    \triangleq & \tr{`m`a.[`b](M[`b/`g])},                     & \square, & \gp:[]
\end{array}
\]
\\ 
% TODO: verify that this is correct
\[
\begin{array}{rlrl}
  \prooflabel{(`m`d.[`a]M)[[`a]M^\prime N/[`a]M^\prime]
    \to (`m`d.[`a](M[[`a]M^\prime N/[`a]M^\prime])N} \\
               & \tr{(`m`d.[`a]M)[[`g]M^\prime N/[`a]M^\prime]}\\
    \triangleq & (\wsc\ \gp\ `l`d.\pp\ \gp\ (\psc\ `a\ M))[\square N/`a]         & \square, & \gp:[]  \\
    \tob\      & \wsc\ \gp\ `l`d.\pp\ \gp\ (\psc\ \square N\ (M[\square N/`a]))) & \square, & \gp:[]  \\
    \triangleq & \tr{`m`d.[`a](M[[`a]M^\prime N/[`a]M^\prime])N}
\end{array}
\]
\\ 
% TODO: verify that this is correct
\[
\begin{array}{rlrl}
  \prooflabel{M[[`a]M^\prime N/[`a]M] \to M \hspace{10pt} (`a \not\in \fn(M)):} \\
               & \tr{M[[`a]M^\prime N/[`a]M]} \\
    \triangleq & \tr{M}[\square N/`a] \\
    \tob\      & \tr{M} & (`a \not\in \fv(M))
\end{array}
\]
\end{proof}

This means that the translation of $M$ reduces a term $Q$ that relates to the untranslated reduct of $N$, given $M \to N$.
The relation is either that $Q$ reduces to $N$, is the same as $N$, or $N$ reduces to $Q$.

\begin{theorem}[Completeness of $\tr{\bullet}$]

\end{theorem}

We attempt to prove one of the following properties:
\begin{enumerate}
  \item $\tr{M} \to Q \Rightarrow \exists N. M \to N \land Q \to^* \tr{N}$
  \item $\tr{M} \to^{nf} Q \Rightarrow \exists N. M \to^* N \land \tr{N} = Q$
  \item $\tr{M} \to^{nf} Q \Rightarrow \exists N. M \to^* N \land \tr{N} \to^{nf} Q$
\end{enumerate}

\section{Interpreting \ltry\ in DCC}

  By appending the interpretation of \ltry\ in $`l`m$ with the interpretation of
  $`l`m$ in DCC, we get a translation from \ltry\ to DCC:

  % TODO: add lines over (catch n(x) = L)
  \definition{
    \textsc{Translation of \ltry\ into DCC}
    \[
    \begin{array}{rcl}
      \tr{x} & \triangleq & x \\
      \tr{`lx.M} & \triangleq & `lx.\tr{M} \\
      \tr{MN} & \triangleq & \tr{M} \tr{N} \\
      
      \tr{\throw{n($M$)}} & \triangleq & \wsc\ \gp\ `l\nonocc.\pp\ \gp\ (psc\ n\ (c_n\ \tr{M})) \\ 
      \tr{\try M;\ \catch{n($x$) = $L$}}
        &
        \triangleq 
        &
        (`lc_n.\wsc\ \gp\ `ln.\pp\ \gp\ (\psc\ n\ \tr{M}))(`lx.\tr{L})
      \\ 
      %\multicolumn{3}{l}{\tr{\try M;\ \catch{n$_i$($x$) = $M_i$};\ \catch{m($x$) = $L$}}} \\
      \multicolumn{3}{l}{\tr{\try M;\ \mcatch;\ \catch{m($x$) = $L$}}} \\
        & \triangleq \\
      \multicolumn{3}{r}{
        %(`lc.\wsc\ \gp\ `lm.\pp\ \gp\ (\psc\ m\ \tr{\try M;\ \catch{n$_i$($x$) = $M_i$}}))(`lx.\tr{L})
        (`lc_m.\wsc\ \gp\ `lm.\pp\ \gp\ (\psc\ m\ \tr{\try M;\ \mcatch}))(`lx.\tr{L})
      }
    \end{array} 
    \]
  }

% TODO: describe the interpretation

%\chapter{Conclusion}

This chapter evaluates parts of the methodology and findings of the project.
With specific reference to individual components, it examines what should have been done differently.
It closes with a presentation of suitable directions for future work, 
especially regarding research questions thrown up and still unanswered by the present work.

\subsection{Evaluation}
\subsubsection{Interpreter}
We often treat terms generically: we are not specifically interested in the form of the term.
For example the $M$ and $N$ in the term $M[N/x]$ are generic.
Generic terms like this were used extensively in the soundness and completeness proofs for the \lmu\ translation.
A minor extension to the interpreter would be to extend it with the constants.
Like a value, the evaluation of a constant would leave the constant the same. 
This extension would allow the expression of these generic terms.
In this way, the output of the interpreter could have been used to produce proofs without requiring amendments.
Additionally, 
the interpreter could have output \LaTeX\ by writing an appropriate function of type \mono{State -> String} to pass into the evaluation engine.

\subsubsection{Choice of Calculus}
CDC was not the correct choice of calculus to facilitate a translation from \ltry\ to Haskell.  
It was chosen because it easily facilitated a translation to Haskell: 
the syntax of CDC was intentionally close to Haskell
and a CDC library had been implemented in Haskell by the authors of the calculus.
However there is an impedance mismatch between CDC and \lmu.
The translation does not use most of the expressive power of CDC.
For example, 
CDC is capable of generating multiple prompts of different names.
Only a single prompt is used by the translation of the \lmu-calculus. 
We do not need a continuation stack or multiple named prompts, we just need a single delimiter.
We could get away with a map from prompts to contexts in place of a stack.
This would have greatly simplified the translation although it would also require reimplementing the CDC operators.
We believe this would be useful to develop in follow up work.

\subsubsection{Direct Translation}
There is possible a more direct translation of \ltry\ to CDC that is not mediated by a translation to \lmu.
More importantly, there is likely to be a more direct implementation of \ltry\ to Haskell.
By translating through two other calculi, we have left this possibility unexamined.
Although we still would want a direct implementation of \ltry\ in Haskell that does not use CDC,
we have demonstrated that an implementation is possible and laid down the ground work for this to be done.

%\subsubsection{Types}
%We did not investigate whether types were preserved by the translation.
%
%The typing of CDC terms is left to Haskell.
%A formal treatment would have been better.

\subsubsection{Translation and Soundness}
The embedding of the \lmu-calculus in CDC is not as close to the \lmu-calculus as it could be.
There are some reduction steps in the destination language that are not reflected by the source language.
Specifically, the soundness property we proved was
\[
  M \to_{`m} N \Rightarrow \exists P. \tr{M} \to^{*} P \land \tr{N} \to^{*} P
\]
This says that the translation of $M$ and the translation of the $`m$-reduction of $M$ reduce to the same term.
By defining the substitition of terms explicitly, as in \cite{Bakel14}, we may have proved
\[
  M \to_{`m}^{*} N \Rightarrow \tr{M} \to^{*} \tr{N}
\]
This says that the translation of $M$ reduces to the translation of $N$.
With this property, the correspondence between the source language and the target language would have be closer.

\subsection{Conclusion}
We defined an interpreter for CDC which we used to experiment with translations from the \lmu-calculus.
The output of the interpreter was transliterated into derivations that we used to prove the soundness and completeness of our translation.
Using this translation, along with van Bakel's \ltry-to-\lmu\ translation, we defined a translation from \ltry\ to CDC.
This was used as the basis for some proof-of-concept \ltry\ implementations in Haskell.
This project presents a clear demonstration that named exceptions, as introduced by the \ltry-calculus, are implementable in Haskell.
Apart from a minor syntactic extension,
the implementation would not require any machinery currently unavailable to Haskell.
    
\subsubsection{Future Work}
Finally, we present a list of suitable directions for future work.
\begin{itemize}
  \item \emph{Type preservation} -- Both \ltry\ and \lmu\ have type assignment systems. We did not explore whether the types are preserved under the translations we defined. Doing this is will give additional insight into how the \ltry\ calculus relates to delimited continuations and how it can be implemented in Haskell without using CDC.
  \item \emph{Haskell extension} -- Proof-of-concept Haskell libraries are presented and a language extension is proposed but the language extension is not implemented. Given the state of the present work, this will be trivial. By exposing the functionality in a language pragma, we will extend the parser with additional grammar rules.
  \item \emph{Reimplement CDC without stack} -- The implementation of \lmu\ in CDC only ever uses a single prompt: CDC is overengineered for the purposes of a \lmu-translation. We will rewrite CDC to replace the context and prompt stack with a mapping from prompts to contexts. Using this, a simpler translation of \lmu\ to CDC should be found.
  \item \emph{Translate \ltry\ directly into Haskell} -- The current methodology helped expand our understanding of the relation of \ltry\ to Haskell. Using this, we can write a direct implementation of \ltry\ into Haskell that bypasses CDC entirely.
\end{itemize}

%
%\newpage
%\bibliographystyle{plain}
%\bibliography{references}

\end{document}
