\documentclass[11pt,a4paper,twoside,draft]{report}

\usepackage{scalerel}
\usepackage{titlesec}
\usepackage{graphicx}
\usepackage{amsthm}
\usepackage{qsymbols}
\usepackage{listings}
\usepackage{amsmath}
\usepackage{amssymb}
\usepackage{derivation}
\usepackage{theorems}
\usepackage{relsize}
\usepackage{fancyvrb}
\usepackage{verbatim}
\usepackage{pstricks}
\usepackage{proof}
\usepackage{float}

\usepackage{macros}

\newcommand{\reporttitle}{Exception Handling in Haskell Using the \ltry-Calculus}
\newcommand{\name}{William S. Fisher}
\newcommand{\supervisor}{Steffen van Bakel}
\newcommand{\course}{Computing Science}


\definecolor{pale-gray}{gray}{0.75}
\newcommand{\hsp}{\hspace{15pt}}
\titleformat{\chapter}[hang]{\Huge\bfseries}{\thechapter\hsp\color{pale-gray}{|}\hsp}{0pt}{\Huge\bfseries}

\begin{document}

%\begin{titlepage}

\center

\textsc{\Large{Imperial College London}}\\[0.5cm] 
\textsc{\large{Department of Computing}}\\[0.5cm] 

\vspace{4cm}
\huge{\textbf{\reporttitle}}
\vspace{2cm}

\large{
  \emph{by} \\
  \vspace{0.1cm}
  \textbf{\name}
}

\vspace{1cm}

\large{
  \emph{supervised by} \\
  \vspace{0.1cm}
  \textbf{\supervisor}
}

\vfill % Fill the rest of the page with whitespace
\large{
Submitted in partial fulfillment of the requirements for the MSc degree in
\course~of Imperial College London\\[0.5cm]
}

\makeatletter
\@date 
\makeatother

\end{titlepage}

%\begin{abstract}
%We present a language-level extension to Haskell for exception handling.
%We outline a series of translations between formal systems, starting from the \ltry-calculus.
%Unlike other approaches in either software or formal systems, the \ltry-calculus uses named exception handlers.
%We define a term-rewriting interpreter for the Calculus of Delimed Continuations in Haskell.
%This interpreter produces the reduction steps required to prove the soundness and completeness of our translations.
%Finally, we explore both a proof-of-concept, library-level implementation and a language-level extension for Haskell that are based on the \ltry-calculus.
%\end{abstract}
%
%\pagenumbering{gobble}
%
%\chapter*{Acknowledgements}





%
%\tableofcontents
%\clearpage{\pagestyle{empty}\cleardoublepage}
%
%\pagenumbering{arabic}
%\setcounter{page}{1}
%
%\chapter{Introduction}

In 1936 Alonzo Church presented the \lam-calculus, a formal system for expressing the entire set of computable problems \cite{Church36}.\footnote{
  The ability of a system to express the entire set of computable problems is now commonly known as `Turing completeness' although here this description would be anachronistic; 
  Turing's work on computability was published a year after Church's.
}
Later, Haskell Curry and William Howard both discover a correspondence between the simply-typed \lam-calculus and intuitionistic logic. 
However the correspondence runs deeper than this: there is actually a true isomorphism between many systems of formal logic and computational calculi. 

The correspondence between systems of formal logic and computer programs drove research into both the logical counterparts of programs and the computational counterparts of logical systems.
As a part of this, Michel Parigot developed the \lmu-calculus, an extension of the \lam-calculus, 
through an isomorphism to classical logic \cite{Parigot92}.
More recently, Steffen van Bakel modelled exceptions and exception handling through an extension of the \lam-calculus called the \ltry-calculus \cite{Bakel15}.\footnote{
  Exceptions are a form of error generated by computers to indicate that the flow of control cannot continue as expected and must be aborted. 
}

Exceptions are most commonly found in two kinds: 
  systems where there is a single type of exception such as in Javascript, 
  and systems where exceptions are discriminated by \emph{type} as in Java and Haskell. 
Unlike either of these approaches, 
the exception-handling proposed by the \ltry-calculus introduces exceptions discriminated by name.

\subsubsection{Project}
This paper proposes an implementation of named-exceptions in Haskell.
First we define a translation from the \lmu-calculus to a calculus for expressing delimited continuations, CDC \cite{JonesDS07}
\[
\begin{array}{lcr}
  `l`m & \to & CDC
\end{array}
\]
CDC is a calculus which closely models Haskell's syntax and which has been extended with operations for manipulating delimited continuations.
The CDC library for Haskell defined by Jones \emph{et al.} in \cite{JonesDS07} evaluates CDC terms in a single step, returning the final value of the reduction.
However, we are interested in producing the entire step-by-step evaluation of CDC terms which we can transliterate into CDC derivations for verifying translations of \lmu\ into CDC.
This motivates us to write a term-rewriting program for evaluating CDC terms.
The proofs generated by this program are used to investigate the soundness and completeness of the $`l`m \to CDC$ translation.
The translation is then concatenated with van Bakel's original translation of his \ltry-calculus into the \lmu-calculus which yields a translation from \ltry\ to CDC:
\[
   (`l^{try} \to `l`m) \circ (`l`m \to CDC) = `l^{try} \to CDC
\]
We use this final translation to explore implementations of named exceptions directly in Haskell.

\subsubsection{Outline}
Chapter \ref{chapter:background} discusses the related context for this project. Following this, this paper makes the following original contributions:
\begin{itemize}
\item A Haskell interpreter for the calculus of delimited continuations, CDC (in Chapter \ref{chapter:interpreter})
\item A translation of \lmu\ into CDC along with proofs of its soundness and completeness with respect to $`m$-reduction (in Chapter \ref{chapter:translations})
\item A translation of \ltry\ into CDC (also in Chapter \ref{chapter:translations}) 
\item A proof of concept implementation of \ltry\ in Haskell, based on this translation (in Chapter \ref{chapter:implementation})
\item The specification for a language extension for Haskell with named exceptions, based on \ltry\ (also Chapter \ref{chapter:implementation})
\end{itemize}

%% TODO: introduce notation and concept of fv/bv and fn/bn

\chapter{Background}

This chapter explores what \emph{formal systems} are and what they are useful for. It looks at a number of related formal systems and their relation to computation. It outlines the context ontop of which the rest of this project is built.

% TODO: explain that ideas will become clearer as they are
%       used and we see how they interact with other concepts
%       introduced later. it's ok to leave concepts not entirely
%       developed

\section{Formal Systems}

Formal systems are a set of rules for writing and manipulating
formulae. Formulae are constructed from a set of characters called the
\emph{alphabet} by following a some formula-construction rules called
the \emph{grammar}. The only formulae considered \emph{well-formed} in
a system are those constructed according to the grammar of a system.
Formal systems are used to model domains of knowledge to help better
and more formally understand those domains.

\subsection{Syntax and Grammars}

The grammar of a formal system describes the system's syntax. Grammars are 
rules for constructing formulae that are well-formed. Formulae
produced according to the grammar of a system are well-formed according to 
the syntax of that system.

We defined grammars using Backus-Naur Form or BNF:

\[
  M ::= t\ \mid\ f
\]

This grammar describes that syntactically-valid constructs are either
the letter $t$ or the letter $f$. Grammars can be recursive
which allows much more expressive construction rules:

\begin{figure}[!h]\label{fig:tf-grammar}
\[
  \begin{array}{lclr}
    M,N&::=&t&\textcolor{pale-gray}{(1)} \\
      &|&f&\textcolor{pale-gray}{(2)} \\
      &|&M\ a\ N&\textcolor{pale-gray}{(3)} \\
      &|&M\ o\ N&\textcolor{pale-gray}{(4)}
  \end{array}
\]
\caption{Grammar for producing the letters $t$ or $f$ connected by the letters $a$ or $o$}
\end{figure}

This grammar describes formulae containing the any number of occurrences of
the letters $t$ or $f$ separated by either an $a$ or an $o$.

\[
\begin{array}{lr}
  \by{t}{1} \\
  \by{t\ o\ f}{2 \& 4} \\
  \by{t\ o\ f\ a\ t}{1 \& 3}
\end{array}
\]

\subsection{Derivation Rules}
Whereas a grammar describes the rules for producing well-formed formulae,
the derivation rules describe rules for transforming formulae of a
particular form into a new formula. Using the grammar from Figure 
\ref{fig:tf-grammar}, we add derivation rules:

\[
\begin{array}{lcl}
  t\ a\ M &\to& M \\ 
  M\ a\ t &\to& M \\
  f\ a\ f &\to& f \\
  \\
  M\ o\ t &\to& t \\ 
  t\ o\ M &\to& t \\
  f\ o\ f &\to& f \\
\end{array}
\]

These rules describe that if a formula matches the pattern on the 
left-hand side, where $M$ represents a well-formed formula, it can be
replaced by the formula on the right-hand side.

\subsection{Domain Modelling}

The syntax and derivation rules of a formal system are defined to model 
some domain. This isomorphism between the domain and the formal system
means we can attempt to discover truths about the domain through studying
the formal system.

For example, take the $tf$-system described above. Without
understanding the domain, we are able to manipulate formulae of the
system to create new formulae. The $tf$-system is isomorphic to
Boolean algebra:

\[
\begin{array}{cc}
\text{$tf$-system} & \text{Boolean algebra} \\
t & 1 \\
t & 0 \\
a & \wedge \\
o & \vee 
\end{array}
\]

Using formal systems allows us to understand the domains they model from
different perspectives and thereby learn novel truths about them.

%\subsection{Derivation Strategies}
% TODO: explain that when you can make more than one decision,
%       you can have a strategy for which decision you take

\section{\lam-Calculus}

In response to Hilbert's \emph{Entscheidungsproblem}, Alonzo Church
defined the \lam-calculus. It is a formal system capable of expressing
the set of effectively-computible algorithms. Ontop of this, he built
his proof that not all algorithms are decidable. Shortly after, Godel
and Turing created their own models of effective computibility.
\footnote{General recursive functions and Turing machines, respectively} 
These models were later proved to all be equivalent.

\subsection{Syntax}
  
  \lam-variables are represented by $x,y,z,\&c$. Variables denote an
  arbitrary value: they do not describe what the value is but that any
  two occurrences of the same variable represent the same value. The
  grammar for constructing well-formed \lam-terms is:

  \begin{figure}[!h]
  \definition{ 
    \textsc{(Grammar for untyped \lam-calculus)}
    \item $`l$-variables are denoted by $x, y,\dots$ \\
    \[
    \begin{array}{rcl}
    M,N & ::= & x\ \mid\ `lx.M\ \mid\ M\ N
    \end{array}
    \]
  }
  \end{figure}

  \lam-abstractions are represented by $`lx.M$ where $x$ is a
  parameter and $M$ is the body of the abstraction. The same idea is
  expressed by more conventional notation as a mathematical function
  $f(x) = M$. The $`l$ annotates the beginning of an abstraction and
  the $.$ separates the parameter from the body of the abstraction.
  This grammar is recursive meaning the body of an abstraction is just
  another term constructed according to the grammar. Some examples of
  abstractions are:
  
  \begin{figure}[!h]
    \[
      \begin{array}{l}
      `lx.x \\
      `lx.xy \\
      `lx.(`ly.xy)
      \end{array}
    \]
  \caption{Examples of valid \lam-abstractions}
  \end{figure}
  
  Applications are represented by any two terms, constructed according to
  the grammar, placed alongside one another. Application gives highest
  precedence to the left-most terms. Bracketing can be introduced to enforce 
  alternative application order for example $xyz$ is implicitly read as 
  $(xy)z$ but an be written as $x(yz)$ to describe that the application of 
  $yz$ should come first. Examples of applications are:
    \begin{figure}[!h]
      \[
        \begin{array}{l}
        xy \\
        xyz \\
        x(yz) \\
        (`lx.x)y
        \end{array}
      \]
    \caption{Examples of valid applications}
    \end{figure}

\subsection{Reduction Rules}

  First we will introduce the substitution notation $M[N/x]$. This denotes 
  the term M with all occurrences of x replaced by N. The substitution 
  notation is defined inductively as:
   
    \begin{figure}[!h]
    \definition{ 
      \textsc{(Substitution notation for \lam-terms)}
      \[
      \begin{array}{rclr}
      x[y/x] & \rightarrow & y \\
      z[y/x] & \rightarrow & z & (z \neq x) \\
      (`lz.M)[y/x] & \rightarrow & `lz.(M[y/x]) \\
      (M N)[y/x] & \rightarrow & M[y/x] N[y/x]
      \end{array}
      \]
    }
    \end{figure}

\subsubsection{\bta-reduction}
  The main derivation rule of the \lam-calculus is \bta-reduction. If term
  $M$ \bta-reduces to term $N$, we write $M \rightarrow_{`b} N$ although 
  the \bta\ subscript can be omitted if it is clear from context. \bta-reduction
  is defined for the application of two terms:
  \begin{figure}[!h]\label{def:beta-reduction}
  \definition{ 
    \textsc{($`b$-reduction for \lam-calculus)}
    \[
    \begin{array}{rcl}
    (`lx.M) N & \rightarrow_{`b} & M[N/x]
    \end{array}
    \]
  }
  \end{figure}
 
  \lam-variables and \lam-abstactions are \emph{values}: they do not reduce
  to other terms. If a formula is a value, the reduction terminates on that 
  value. Only applications reduce to other terms. This means that an 
  application is a reducible expression or a \emph{redex}. Reducing a 
  redex models the computing of a function. 

% TODO: define free variables and bound variables
\subsubsection{$\alpha$-reduction}
  
  The \lam-calculus defines a reduction rule for renaming variables.
  Variable names are arbitrary and chosen just to denote identity:
  all occurrences of $x$ are the same. This can become a problem
  in the following case:
  
  \[
    (`ly.`lx.xy)(`lx.x)
  \]
 
  After the application is reduced, we have the term:
  
  \[
    `lx.x(`lx.x) 
  \]
  
  In this case, it is ambiguous which \lam-abstraction the right-most $x$
  is bound by. When this term is applied to another, will the substitution
  occur to all occurrences of $x$? From the initial term, it is clear that
  this would be incorrect. The \lam-calculus introduces $`a$-reduction to
  solve this:
  
  \begin{figure}[!h]\label{def:alpha-reduction}
  \definition{ 
    \textsc{($`a$-reduction for \lam-calculus)}
  \[
    \begin{array}{rl}
    `lx.M \rightarrow_{`a} `ly.M[y/x] & (y \notin fv(M))
    \end{array}
  \]
  }
  \end{figure}
  
  This means we can rename the lead variable of an abstraction $M$ on the 
  conditions that: 
  \begin{enumerate}
    \item All variables bound by that abstraction are renamed the same 
    \item The variable it is changed to is not currently in use in $M$ 
  \end{enumerate}
  
  
  %TODO: add alpha reduction(?)
  % \subsection{Reduction Strategies}

\section{Logic and Types}
%  \subsection{Logic Systems}
  \subsection{Implicative Intuitionistic Logic}
  A formal system based on Gerhard Gentzen's natural deduction.
  Restricted only to $\rightarrow \mathcal{I}$ and $\rightarrow 
  \mathcal{E}$.
  Following Brouwer, also leaves out law of excluded middle.
  
  \[
  \begin{array}{rl@{\quad\quad}rl}
    ( \to \mathcal{I} ) &
    \Inf { A \to B \hspace{0.5cm} A }
        {B}
    &
    ( \to \mathcal{E} ) &
    \Inf { [A] \vdash B }
        { A \to B}
  \end{array}
  \]
  
%  \subsubsection{Classical}
%  \subsubsection{Sequent}
%
  \subsection{Type Assignment}
  
  Type assignment introduces additional grammar and restrictions on
  the reduction rules of a system. These extensions prevent logically
  inconsistent terms from being constructed. A type assignment has the
  form:
  
  \[
    M : `a 
  \]
  
  which states that term $M$ has the type $`a$. Like variables, type 
  variables are abstract: they do not describe anything more about a 
  type than its identity. That is to say $x: A$ and $y : A$ have the
  same type but we cannot say any more about what that type is.
 
  A type is either some uppercase Latin letter or it is two valid types
  connected by a $\rightarrow$. This is described by the following
  BNF grammar:
  
  \[
    A,B ::= `v \mid A \rightarrow B 
  \]
  
  Typing rules have their own form that resembles Gentzen's sequent 
  calculus:
  
  \[
    `G \vdash T 
  \]
    
  This describes that $`G$ is a set of typed \lam-terms and $T$ is a
  typed \lam-term that is derivable from $`G$.

  \subsection{Curry-Howard Isomophism}
  
  Curry-Howard isomorphism states that their is an isomorphism between
  the typed of a term and a logical proposition. The term itself is
  the proof of the proposition.
  
  % TODO: add more - examples and demonstration

\section{Haskell}
  % TODO: describe purity and laziness
  
  \subsection{Data Types}
  New data types can be introduced into Haskell in 3 distinct ways. First,
  using the \mono{data} keyword:
  
  \Verbatimcode
    data Animal a = Dog a
      | Cat
  \end{Verbatim}
  
  The \mono{data} keyword begins the definition of a new data type. The
  word immediately following determines the type constructor for the new
  type. Following this is a type parameter for the type constructor. There
  can be any number of type parameters, including zero. The right-hand
  side of the \mono{=} introduces a \mono{|}-separated list of data 
  constructors.
  
  \Verbatimcode
    > let hector = Cat
    > :t hector
    hector :: Animal a
    
    > let topaz = Dog "foo"
    > :t topaz
    topaz :: Animal String
  \end{Verbatim}
  
  The type parameter is constrained by the type of the value the data
  constructor was initialized with. In the example above, calling the
  \mono{Dog} data constructor with a string makes the type \mono{Animal
  String} rather than the more general \mono{Animal a}.
  
  The second method for introducing new data types is the \mono{newtype}
  keyword. The key difference between \mono{data} and \mono{newtype} is
  that \mono{newtype} can only have one data constructor. Informally,
  this implies a kind of isomorphism:

  \Verbatimcode
    newtype Foo a = Foo (a -> Integer)
  \end{Verbatim}
  
  The type constructor can take type parameters which will be constrained
  by the inhabitants of the data constructor. This data type expresses
  an isomorphism between \mono{Foo a} and functions from \mono{a} to
  \mono{Integer}s.
  
  Finally, we can introduce type aliases using the \mono{type} keyword:
  \Verbatimcode
    type Name = String 
  \end{Verbatim}
  
  Again, we introduce a type constructor \mono{Name} but this time we
  name another type, in this case \mono{String}, as its inhabitant. This
  means that the type \mono{Name} is a type alias for \mono{String} and
  will share the same data constructors. 
 
  \subsection{Type Level/Value Level}
  Haskell distinguishes between terms on the type level and terms on the
  value level. Type level terms are descriptions of the types of a value.
  They provide restrictions on the construction of invalid terms. For
  instance if we have a function of type \mono{String -> Integer}, we cannot
  apply it to a term of type \mono{Boolean}. The type-checker will throw
  an error before any value-level computation is initiated.
 
  % TODO: better description needed
  The value level is the level on which data is constructed and manipulated.
  The operation \mono{1+1} occurs on the value level. The value level is
  where computation takes place and the type level is where static analysis
  of the program type takes place.
  
  \subsection{Type Classes}
  Haskell adds type classes to the type level. Types can have instances of
  type classes. The most similar concept from Object-Oriented programming
  is \emph{interfaces}.
  
  \Verbatimcode
    class Addable a where
      (add) :: a -> a -> a
  \end{Verbatim}
  
  Type classes are introduced using the \mono{class} keyword. Beneath that
  are the function names and corresponding type-signatures of the functions
  that an instance of a class must implement.
  
  For example, we can create instances of the \mono{Addable} class:
  
  \Verbatimcode
    data Number = One | Two | ThreeOrMore
    
    instance Addable Number where
      add One One = Two 
      add One Two = ThreeOrMore 
      add Two One = ThreeOrMore 
  \end{Verbatim}    
  
  %\subsection{Term Rewriting}
  % explain ($) operator
  % explain monads as generalizations of cps-terms
  % what they encode and why they are useful

% TODO: these explanations of continuations (diff between delim and 
%       undelim) are incorrect 
\section{Continuations}
  
  Compound terms can be decomposed into two seperate parts: a dominant 
  term and a context. The dominant term is the term currently being 
  evaluated. The context is a term with a hole that will be filled with 
  the value the dominant term reduces to.
  
  \begin{figure}[!h]
    \hspace{1cm}Assume that $M \rightarrow_{`b} M^\prime$
    \[
    \begin{array}{lrcl}
    \textit{(Compound term)}&& MN \\
    \textit{(Decompose)}&M && \square N \\
    \textit{(Beta-reduce dominant term)}& M^\prime && \square N \\
    \textit{(Refill hole of context)}&& M^\prime N \\
    \end{array}
    \]
  \caption{Decomposing a term into a dominant term and a context}
  \end{figure}
  
  When $M$ has \bta-reduced to a value -- $M^\prime$ -- then the hole
  of the context $\square N$ is filled to form $M^\prime N$. What the
  dominant term and context are for a given term depends on the
  reduction rules and strategy. The context is what remains to be
  reduced at given moment of reduction. Thus a context is also called
  a \emph{continuation}.
 
  \subsection{Undelimited Continuations} 
 
  For more complex terms, the waiting context will grow as the dominant
  term gets further decomposed:
  
  \begin{figure}[!h]
    \[
    \begin{array}{ll}
      (MM^\prime) M^{\prime\prime} \\
      (MM^\prime) & \square M^{\prime\prime} \\
      M & (\square M^\prime) M^{\prime\prime} \\
    \end{array}
    \]
  \caption{Decomposing a term into multiple contexts}
  \end{figure}

  By amalgamating continuatinos into one big continuation we only have
  two components at point during the reduction: the current dominant term
  and the \emph{current continuation}. 
  
  Assume we have some reduction rules defined for manipulating
  continuations. This model keeps continuations grouped together which
  means these hypothetical reduction rules could manipulate only this
  entire remaining continuation. For this reason, continuations that
  can only be manipulated in their entirety are \textbf{undelimited
  continuations}.

  \subsection{Delimited-Continuations}

  Instead, if we maintain a stack of continuations when
  decomposing complex terms, we can keep continuations separated:
  
  \begin{figure}[!h]
    \[
    \begin{array}{lll}
      (MM^\prime) M^{\prime\prime} \\
      (MM^\prime) & \square M^{\prime\prime} \\
      M & \square M^\prime & \square M^{\prime\prime} \\
    \end{array}
    \]
  \caption{Decomposing a term into multiple contexts}
  \end{figure}

  Here, when a dominant term has been reduced, the reduct is returned
  to its corresponding continuation. This newly joined term then
  becomes the dominant redex. After this new dominant term has been
  reduced, it will be returned to the next waiting continuation, and
  so on. Throughout this process, we maintain each continuation
  separately.
  
  Assume again that we have reduction rules defined for manipulating
  continuations. By keeping continuations separate, this model would 
  allow use to use parts of the stack selectively for instance placing
  delimiters between portions of interest. The increased granularity of 
  control means we can manipulate not just the entire remaining continuation 
  but sections of it. Thus, continuations of this kind are called 
  \textbf{delimited continuations}.

  \subsection{Continuation-Passing Style}
 
  By rewriting \lam-terms, the continuations can be made explicit. All
  terms must be turned into \lam-abstractions of some variable $k$
  where $k$ is the continuation of a term. $k$ is then called on the
  result of the term, triggering the continuation to take control.
  This style of writing \lam-terms is called continuation-passing
  style or CPS.
  
  \definition{
    \textsc{(Translation of standard \lam-terms into CPS)} 
    \[
    \begin{array}{lcl}
      \tr{x}     & = & `lk.kx      \\
      \tr{`lx.M} & = & `lk.k(`lx.\tr{M}) \\
      \tr{M N} & = & `lk.M(`lm.m\tr{N}(`ln.mnk)
    \end{array}
    \]
  }
  
  The term that a CPS program terminates on will be of the form
  $`lk.kM$. In order to extract the value, a \emph{final continuation}
  must be provided. Depending on the context, this could be an identity 
  function $`lx.x$ or a display operation $`lx.\text{display }x$ to
  display the results of the program.

  \begin{figure}[!h]
  \caption{Extracting the final value from a terminated CPS program}
  \[
  \begin{array}{ll}
                & (`lk.kM)(`lx.x) \\
    \rightarrow & (`lx.x)M \\
    \rightarrow & M
  \end{array}
  \]
  \end{figure}
 
  The translation of standard \lam-terms into CPS similarly transforms the 
  \emph{types} of \lam-terms. For example, a term $x : A$ becomes 
  $`lk.kx : (A \to B) \to B$. This type represents a delayed computation:
  a computation that is waiting for a function to continue execution with. 
  In order to resume the computation, the term must be applied to a 
  continuation.
  
  As an example, take the term M where
  
  \[
    M = `lk.kx
  \]

  To access the value $x$ contained in $M$, we have to apply $M$ to a
  continuation function $`lm.N$:
  
  \[
  \begin{array}{l}
      (`lk.kx)(`lm.N) \\
      (`lm.N)x \\
      N[x/m]
  \end{array}
  \]
  
  Within the body of $N$, $m$ is bound to the value contained by $M$.
  So we can think about $M$ as a suspended computation that, when applied
  to a continuation, applies the continuation to $x$. Looking at
  the type $(A \to B) \to B$ again, it is clear that $A$ is the type
  of the term passed to the continuation of a CPS-term:
  
  \[
  \begin{array}{l}
    `lk.kx : (A \to B) \to B \\
    k : (A \to B) \\
    kx : B \\
    x : A \\
  \end{array} 
  \]
  
  % TODO: explain how order-of-evaluation can be enforced using 
  %       cps.
 
  \subsection{Monads}
  
  If we have two suspended computations $M$ and $M^\prime$ and we want to
  run $M$ and then $M^\prime$, we have to apply $M$ to a continuation
  to access its value and then do the same to $M^\prime$:
  
  \[
    M(`lm.M^\prime(`lm^\prime.N))
  \]
  
  This is a common operation so we define a utility operator \mono{>>=}
  that binds the first suspended computation to a continuation which 
  returns another suspended computation\footnote{ \mono{(>>=)} is pronounced 'bind'.}:
 
  \[
    \mono{>>=} : ((A \to B) \to B) \to (A \to ((B \to C) \to C)) \to ((B \to C) \to C)
  \]
 
  The type $(A \to B) \to B$ that represents a suspended computation returning
  a value of type $A$ to its continuation we will call an $A$-computation or
  $Comp\ A$. We can rewrite the type signature of \mono{>>=}:
  
  \[
    \mono{>>=} : Comp\ A \to (A \to Comp\ B) \to Comp\ B
  \]
  
  We define another operator, $return$, that takes a value and returns a 
  suspended computation that returns that value:
  
  \[
    return : A \to Comp\ A 
  \]
 
  The type constructor $Comp\ A$, together with the two utility functions
  \mono{>>=} and $return$, make up Haskell's Monad type class:
  
  \Verbatimcode
    class Monad M where
      (>>=) :: M a -> (a -> M b) -> M b
      return :: a -> M a
  \end{Verbatim}
  
  The Monad type class generalizes CPS terms: they represent suspended
  computations that can be composed using \mono{>>=}. Just like CPS terms,
  a Monad type \mono{M a} tells us that we have a term that will pass
  values of type \mono{a} to the continuation it is bound to using
  \mono{>>=}.
  
  % TODO: finish explanation of linking suspended-computations and CPS
  %       to monads

\section{\lmu-Calculus}
  % TODO: intuitively, that mu-terms are expressed easily by continuation 
  %       passing style gives us some idea that they are about control flow
  %       and that they will be easily expressed by monads
  \subsection{Syntax}
  \begin{figure}[!h]
  \definition{ 
    \textsc{(Grammar for \lmu-calculus)}
    \item $`l$-variables are denoted by $x, y,\dots$ and $`m$-variables are denoted by $`a, `b,\dots$ \\
    \[
    \begin{array}{lrcl}
    
    \text{(Unnamed term)} & M,N & ::= & x\ |\ `lx.M\ |\ M\ N\ |\ `m`a.C \\
    \text{(Named term)} & C & ::= & [`a]M
    \end{array}
    \]
  }
  \end{figure}
  
  Just as \lam\ introduces a new \lam-abstraction, \lmu\ introduces a new 
  \lmu-abstraction. The body of a \lmu-abstraction must be a named term. 
  A named term consists of a name of the form $[`a]$ followed by an unnamed 
  term. 

  \subsection{Reduction Rules}
  \begin{figure}[!h]
  \definition{ 
    \textsc{(Reduction rules for \lmu-calculus)}
    \[
    \begin{array}{rcl}
    x & \rightarrow & x \\
    `lx.M & \rightarrow & `lx.M \\
    `m`a.[`b]M & \rightarrow & `m`a.[`b]M \\
    (`lx.M) N & \rightarrow & M[N/x] \\
    (`m`a.[`b]M) N & \rightarrow & (`m`a.[`b]M[[`g]M^\prime N/[`a]M^\prime]) \\
    \end{array}
    \]
  }
  \end{figure}

  The terse reduction rule at the end simple states that the application
  of a \lmu-abstraction $`m`a.M$ to a term $N$ applies all the sub-terms 
  of $M$ labelled $[`a]$ to $N$ and relabels them with a fresh $`m$ 
  variable.
  
  \subsection{Computational Significance}

  % TODO: use cps translation to explain mu

  The additional \lmu\ reduction rules model context manipulation. 
  \lmu-variables map to contexts. When an unnamed term is labelled with a 
  \lmu-variable, it is evaluated in that context. For instance the named 
  term $['a]M$ has the effect of evaluating $M$ in the context pointed to 
  by $`a$.
  
  To make this more concrete, consider the compound term $(`m`a.[`b]M)\ N$. 
  First we decompose the term into a dominant term $(`m`a.[`b]M)$ and a 
  context $\square N$. Informally, we can imagine that the \lmu-variable 
  $`a$ now maps to this context $\{`a \Rightarrow \square N\}$:
  
  % TODO: reform to get rid of context mapping:
  %   keep mu abstraction in dominant and consult context for
  %   what to apply named-terms to
  \begin{example}[]
    \[
    \begin{array}{lc}
    \textbf{Dominant} & \textbf{Context} \\
    (`m`a.[`b]M) N \\
    `m`a.[`b]M & \square N \\
    \end{array}
    \]
  \end{example}

  All subterms of $M$ labelled $`a$ will now be evaluated in the context 
  $\square N$ and the context will destroyed. For example, let us replace 
  $M$ with \mbox{$`m\circ.[`a](`ls.fs)$}:
  
  \begin{example}
    \[
    \begin{array}{lcr}
    \textbf{Dominant} & \textbf{Context} \\
    `m`a.[`b]`m\circ.[`a](`ls.fs)    & \square N \\
    `m`a.[`a](`ls.fs)    & \square N \\
    `m`g.[`g](`ls.fs)N   & & (`g\ \text{fresh})  \\
    \end{array}
    \]
  \end{example}

  % TODO: explain applicative contexts
  After applying the term $[`a](`ls.fs)$ to $N$, the context $\square N$
  is consumed and every occurrence of $`a$ is replaced with a fresh 
  variable -- in this case a $`g$ -- to clarify that the new 
  \lmu-abstraction points to a new context. This means that 
  \lmu-abstractions will pass all of the applicative contexts to the
  named subterms:
  
  \begin{example}
    \[
    \begin{array}{lcr}
    \textbf{Dominant} & \textbf{Context} \\
    (`m`a.[`a](`ls.`lt.st)) M N \\
    (`m`a.[`a](`ls.`lt.st))M & \square N \\
    `m`a.[`a](`ls.`lt.st) & \square M:\square N \\
    `m`g.[`g](`ls.`lt.st)M & \square N & (`g\ \text{fresh}) \\
    `m`d.[`d](`ls.`lt.st)MN & \square N & (`d\ \text{fresh}) \\
    \end{array}
    \]
  \end{example}
  
  % \subsection{Reduction Strategies}
  \subsection{Isomorphism \& Computational Interpretation}

\section{\ltry-Calculus}

\section{Delimited-Continuation Calculus}

  Simon Peyton-Jones \textit{et al.}\ extended the \lam-calculus with additional operators in order create a framework for implementing delimited continuations \cite{JonesDS07}. This calculus will be referred to as the delimited-continuation calculus or DCC. Many calculi have been devised with control mechanisms. Like the \lmu-calculus, these control mechanisms are all specific instances of delimited and undelimited continuations. DCC provides a set of operations that are capable of expressing many of these other common control mechanisms.

  The grammar of DCC is an extension of the standard \lam-calculus:

  \subsection{Syntax}
  \begin{figure}[!h]
  \definition{ 
    \textsc{(Grammar for DCC)}
    \[
    \begin{array}{lrcl}
    \textrm{(Variables)} & x, y, \dots \\
    \textrm{(Expressions)} & e & ::= & x\ |\ `lx.e\ |\ e\ e^\prime \\
                           &   &  |  &  newPrompt\ |\ pushPrompt\ e\ e \\
                           &   &  |  &  withSubCont\ e\ e\ |\ pushSubCont\ e\ e
    \end{array}
    \]
  }
  \end{figure}

  \subsection{Reduction Rules}
  The operational semantics can be understood through an abstract machine that transforms tuple of the form $\langle e,\ D,\ E\, q \rangle$:

  \begin{figure}[!h]
  \relscale{0.9}
  \definition{ 
    \textsc{(Operational semantics for DCC)}
    \[
    \begin{array}{lrcl}
      \langle e\ e^\prime, D, E, q \rangle &\Rightarrow &\langle e, D[\square\ e^\prime], E, q \rangle &\text{e non-value} \\
      \langle v\ e, D, E, q \rangle &\Rightarrow &\langle e, D[v\ \square], E, q \rangle &\text{e non-value} \\
      \langle pushPrompt\ e\ e^\prime, D, E, q \rangle &\Rightarrow &\langle e, D[pushPrompt\ \square\ e^\prime], E, q \rangle &\text{e non-value} \\
      \langle withSubCont\ e\ e^\prime, D, E, q \rangle &\Rightarrow &\langle e, D[withSubCont\ \square\ e^\prime], E, q \rangle &\text{e non-value} \\
      \langle withSubCont\ p\ e, D, E, q \rangle &\Rightarrow &\langle e, D[withSubCont\ p\ \square], E, q \rangle &\text{e non-value} \\
      \langle pushSubCont\ e\ e^\prime, D, E, q \rangle &\Rightarrow &\langle e, D[pushSubCont\ \square\ e^\prime], E, q \rangle &\text{e non-value} \\
    \\
      \langle (`lx.e)\ v, D, E, q \rangle &\Rightarrow &\langle e[v/x], D, E, q \rangle \\
      \langle newPrompt, D, E, q \rangle &\Rightarrow &\langle q, D, E, q+1 \rangle \\
      \langle pushPrompt\ p\ e, D, E, q \rangle &\Rightarrow &\langle e, \square, p : D : E, q \rangle \\
      \langle withSubCont \ p\ v, D, E, q \rangle &\Rightarrow &\langle v (D : E\textsmaller[1]{\overset{p}{\uparrow}}, \square, E\textsmaller[1]{\overset{p}{\downarrow}}, q \rangle \\
      \langle pushSubCont E^\prime\ e, D, E, q \rangle &\Rightarrow &\langle e, \square, E^\prime +{+} (D : E), q \rangle \\
    \\
      \langle v, D, E, q \rangle &\Rightarrow &\langle D[v], \square, E, q \rangle \\
      \langle v, \square, p : E, q \rangle &\Rightarrow &\langle v, \square, E, q \rangle \\
      \langle v, \square, D : E, q \rangle &\Rightarrow &\langle v, D, E, q \rangle
    \end{array}
    \]
  }
  \end{figure}
  
  \subsection{Significance}

  % TODO: ensure prompts and continuation stack has been explained before reaching this point
  The additional terms behave as follows:
  \begin{itemize}
  \item \op{newPrompt} returns a new and distinct prompt.
  \item \op{pushPrompt}'s first argument is a prompt which is pushed onto the continuation stack before evaluating its second argument. 
  \item \op{withSubCont} captures the subcontinuation from the most recent occurrence of the first argument (a prompt) on the excution stack to the current point of execution. Aborts this continuation and applies the second argument (a \lam-abstraction) to the captured continuation.
  \item \op{pushSubCont} pushes the current continuation and then its first argument (a subcontinuation) onto the continuation stack before evaluating its second argument.
  \end{itemize}

\chapter{CDC Interpreter}

This chapter explores the implementation of an interpreter for CDC. Portions of source code are examined in detail although the full source can be found in the appendix.

\section{Implementation}

Although Peyton-Jones \emph{et al.} implement a language-level module for CDC, 
we are interested in the intermediate term transformations.
Using the step-by-step transformations produced by this interpreter,
we can construct and verify the implementations of \ltry\ and \lmu\ into CDC.
Examining transformation steps in full also allows us to derive proofs of soundness and completeness for these translations. 
For this reason, the interpreter was implemented as a term-rewriting program.

We follow the operational semantics of the system to provide an implementation.
This is not necessary and results in an ineffecient implementation.
Despite this, it is the simplest approach to implementation and efficiency is not central to producing proofs.

\subsection{Data structures}
There are two data types for representing CDC terms, \mono{Value} and \mono{Expr}
(Figure \ref{fig:cdc-data-structures}).
Values are not evaluated: when a term has been reduced to a value, 
it has terminated on that value. 
An expression (\mono{Expr}) is a term that can be evaluated to another term. 
The only exception is a \mono{Hole} which can take any position an expression can. 
For this reason, it must be a data constructor for expression types.

The core of the abstract machine is a function from one state to the next. 
A state is its own data type which corresponds to the tuple from the specification of the semantics of the abstract machine $\langle e,\ D,\ E,\ q\rangle$.

\begin{figure}\label{fig:cdc-data-structures}
\Verbatimcode
data Value = Var Char
  | Abs Char Expr
  | Prompt Int
  | Seq [Expr]
  deriving (Show, Eq)
  
data Expr = Val Value 
  | App Expr Expr
  | Hole
  | PushPrompt Expr Expr
  | PushSubCont Expr Expr
  | WithSubCont Expr Expr
  | NewPrompt
  
  | Sub Expr Expr Char
  deriving (Show, Eq)

data State = State Expr Expr [Expr] Value
  deriving (Show, Eq)
\end{Verbatim}
\caption{Data structures for the CDC interpreter}
\end{figure}

\subsection{Utility Functions}
Some utility functions are simplify the implementation.
Informally, these functions behave as follows (see Figure ~\ref{fig:cdc-helper-functions} for implementations details):

\begin{itemize}\itemsep0.1cm

\item \mono{prettify :: Expr -> String} is defined inductively for pretty-\\ printing terms.

\item \mono{ret :: Expr -> Expr -> Expr} returns the first expression with any holes filled in by the second expression.

\item \mono{composeContexts :: [Expr] -> Expr} takes a sequence of expressions and, starting from the end, 
fills the hole of each expression with the previous expression. 
This in effect joins the output of each context with the input of the next context.

\item \mono{promptMatch :: Int -> Expr -> Bool} returns true if the second argument is a Prompt with the same value as the first argument

\item \mono{splitBefore :: [Expr] -> Int -> [Expr]} returns the sequence of expressions up until (but not including) the prompt matching the second argument.

\item \mono{splitAfter :: [Expr] -> Int -> [Expr]} returns the sequence of expressions from (but not including) the prompt matching the second argument.

\item \mono{sub :: Expr -> Expr -> Char -> Expr} returns the first expression with all occurences of the third expression replaced by the second expression. 
If we name the arguments \mono{sub M V x} then this corresponds to the result of evaluating the substitution notation $M[v/x]$.
\end{itemize}

\begin{figure}\label{fig:cdc-helper-functions}
\Verbatimcode
  ret :: Expr -> Expr -> Expr
  ret d e = case d of
    Hole -> e
    App m n -> App (ret m e) (ret n e)
    Val (Abs x m) -> Val $ Abs x (ret m e)
    PushPrompt m n -> PushPrompt (ret m e) (ret n e)
    WithSubCont m n -> WithSubCont (ret m e) (ret n e)
    PushSubCont m n -> PushSubCont (ret m e) (ret n e)
    otherwise -> d

  composeContexts :: [Expr] -> Expr
  composeContexts = foldr ret Hole . reverse

  sub :: Expr -> Expr -> Char -> Expr
  sub m v x = case m of
    Val (Var n) -> if n == x then v else m
    Val (Abs y e) -> Val (Abs y $ sub e v x)
    Val (Prompt p) -> Val (Prompt p)
    App e e' -> App (sub e v x) (sub e' v x)
    NewPrompt -> NewPrompt
    PushPrompt e e' -> PushPrompt (sub e v x) (sub e' v x)
    WithSubCont e e' -> WithSubCont (sub e v x) (sub e' v x)
    PushSubCont e e' -> PushSubCont (sub e v x) (sub e' v x)

  promptMatch :: Int -> Expr -> Bool
  promptMatch i p = case p of
    (Val (Prompt p')) -> i == p'
    otherwise -> False
 
  splitBefore :: [Expr] -> [Expr]
  splitBefore p es = takeWhile (not . promptMatch p) es

  splitAfter :: [Expr] -> [Expr]
  splitAfter  p es = case length es of
    0 -> []
    otherwise -> tail list
    where list = dropWhile (not . promptMatch p) es
\end{Verbatim}
\caption{Utility functions for CDC interpreter}
\end{figure}

% TODO: make this section a little more structured
\subsection{Reduction Rules}

The heavy lifting of the interpreter is done by the function \mono{eval :: State -> State}. 
\mono{eval} is defined inductively on the structure of CDC terms. 
Using pattern-matching, each case of \mono{eval} corresponds directly to at least one of the reduction rules of the CDC abstract machine. 

\subsubsection{Application}
{
\relscale{0.9}
\[
\begin{array}{lcll}
\langle e\ e^\prime, D, E, q \rangle &\to &\langle e, D[\square\ e^\prime], E, q \rangle &\text{e non-value} \\
\langle v\ e, D, E, q \rangle &\to &\langle e, D[v\ \square], E, q \rangle &\text{e non-value} \\
\langle (`lx.e)\ v, D, E, q \rangle &\to &\langle e[v/x], D, E, q \rangle \\
\end{array}
\]
}
The \mono{App e e'} case deals with applications:
if both terms are values and the first term is an abstraction of the form \mono{$`l$x.m}, 
the dominant term becomes a substitution of \mono{e'} for \mono{x} in \mono{m}. 
Otherwise, the term that is not a value is made the dominant term and the remainder of the application is added to the current context. 
If both terms are redexes, the left-most is made the dominant term first. 
In effect, an application first ensures the left-hand term has been evaluated fully before evaluating the right-hand term.
\Verbatimcode
eval (State (App e e') d es q) = case e of
  Val v -> case e' of 
    Val _ -> case v of 
      Abs x m -> State (Sub m e' x) d es q
      Seq es' -> State (ret (composeContexts es') e') d es q
      otherwise -> State (App e e') d es q
    otherwise -> State e' (ret d (App e Hole)) es q
  otherwise -> State e (ret d (App Hole e')) es q
\end{Verbatim}

\subsubsection{PushPrompt}
{
\relscale{0.9}
\[
\begin{array}{lcll}
\langle pushPrompt\ e\ e^\prime, D, E, q \rangle &\to &\langle e, D[pushPrompt\ \square\ e^\prime], E, q \rangle &\text{e non-value} \\
\langle pushPrompt\ p\ e, D, E, q \rangle &\to &\langle e, \square, p : D : E, q \rangle \\
\end{array}
\]
}

The \mono{PushPrompt e e'} case ensures the left term is a value.
It then pushes the first argument (a prompt) and the current context onto the stack and makes the second argument the dominant term. 
\Verbatimcode
eval (State (PushPrompt e e') d es q) = case e of
  Val _ -> State e' Hole (e:d:es) q
  otherwise -> case d of
    Hole -> State e (PushPrompt Hole e') es q
    otherwise -> State e (ret d (PushPrompt Hole e')) es q
\end{Verbatim}

\subsubsection{WithSubCont}
{
\relscale{0.9}
\[
\begin{array}{lcll}
\langle withSubCont\ e\ e^\prime, D, E, q \rangle &\to &\langle e, D[withSubCont\ \square\ e^\prime], E, q \rangle &\text{e non-value} \\
\langle withSubCont\ p\ e, D, E, q \rangle &\to &\langle e, D[withSubCont\ p\ \square], E, q \rangle &\text{e non-value} \\
\langle withSubCont \ p\ v, D, E, q \rangle &\to &\langle v (D : E\until{p}, \square, E\from{p}, q \rangle \\
\end{array}
\]
}

The reduction rules for \mono{WithSubCont e e'} ensure that the first argument has been evaluated to a prompt \mono{p} and then that the second argument has been evaluated to an abstraction. 
Finally, it appends the current continuation to the sequence yielded by splitting the continuation stack at \mono{p}, 
and creates an application of the second argument to this sequence.
\Verbatimcode
eval (State (WithSubCont e e') d es q) = 
  case e of
    Val v -> case e' of
      Val _ -> case v of (Prompt p) -> 
        State (App e' (seq' (d:beforeP))) Hole afterP q
          where beforeP = splitBefore p es
                afterP = splitAfter p es
      otherwise -> State e' (ret d (WithSubCont e Hole)) es q 
    otherwise -> State e (ret d (WithSubCont Hole e')) es q 
\end{Verbatim}

\subsubsection{PushSubCont}
{
\relscale{0.9}
\[
\begin{array}{lcll}
\langle pushSubCont\ e\ e^\prime, D, E, q \rangle &\to &\langle e, D[pushSubCont\ \square\ e^\prime], E, q \rangle &\text{e non-value} \\
\langle pushSubCont E^\prime\ e, D, E, q \rangle &\to &\langle e, \square, E^\prime \app (D : E), q \rangle \\
\end{array}
\]
}
Reducing \mono{PushSubCont e e'} ensures that the first argument is a sequence.
Then it pushes the current continuation, followed by this sequence, onto the stack.
The second argument is promoted to be the dominant term.
This has the effect of evaluating the dominant term and return the result to the sequence.
\Verbatimcode
eval (State (PushSubCont e e') d es q) = 
  case e of
    Val (Seq es') -> State e' Hole (es'++(d:es)) q
    otherwise -> State e (ret d (PushSubCont Hole e')) es q
\end{Verbatim}

\subsubsection{Substitution}
The reduction of \mono{Sub e y x} uses \mono{sub} to recursively substitute the third argument with the second in the first. 
\Verbatimcode
eval (State (Sub e y x) d es q) = 
  State (sub e y x) d es q
\end{Verbatim}

\subsubsection{NewPrompt}
{
\relscale{0.9}
\[
\begin{array}{lcll}
\langle newPrompt, D, E, q \rangle &\to &\langle q, D, E, q+1 \rangle
\end{array}
\]
}
Evaluating \mono{NewPrompt} places the value of the current prompt as the dominant term and increments the global prompt counter:
\Verbatimcode
eval (State NewPrompt d es (Prompt p)) = 
  State (Val (Prompt p)) d es (Prompt $ p+1)
\end{Verbatim}

%\Verbatimcode
eval :: State -> State
eval (State (App e e') d es q) = case e of
  Val v -> case e' of 
    Val _ -> case v of 
      Abs x m -> State (Sub m e' x) d es q
      Seq es' -> State (ret (composeContexts es') e') d es q
      otherwise -> State (App e e') d es q
    otherwise -> State e' (ret d (App e Hole)) es q
  otherwise -> State e (ret d (App Hole e')) es q

eval (State (PushPrompt e e') d es q) = case e of
  Val _ -> State e' Hole (e:d:es) q
  otherwise -> case d of
    Hole -> State e (PushPrompt Hole e') es q
    otherwise -> State e (ret d (PushPrompt Hole e')) es q

eval (State (WithSubCont e e') d es q) = case e of
  Val v -> case e' of
    Val _ -> case v of (Prompt p) -> State (App e' (seq' (d:beforeP))) 
                                            Hole afterP q
                                     where beforeP = splitBefore p es
                                           afterP = splitAfter p es
    otherwise -> State e' (ret d (WithSubCont e Hole)) es q 
  otherwise -> State e (ret d (WithSubCont Hole e')) es q 
  
eval (State (PushSubCont e e') d es q) = case e of
  Val (Seq es') -> State e' Hole (es'++(d:es)) q
  otherwise -> State e (ret d (PushSubCont Hole e')) es q

eval (State (Sub e y x) d es q) = State e' d es q
  where e' = case e of
          Val (Var m) -> if m == x then y else (Val (Var m))
          Val (Abs h m) -> Val (Abs h (sub m y x))
          App m n -> App (sub m y x) (sub n y x)
          Val (Prompt p) -> Val (Prompt p)
          NewPrompt -> NewPrompt
          PushPrompt e1 e2 -> PushPrompt (sub e1 y x) (sub e2 y x)
          WithSubCont e1 e2 -> WithSubCont (sub e1 y x) (sub e2 y x)
          PushSubCont e1 e2 -> PushSubCont (sub e1 y x) (sub e2 y x)

eval (State NewPrompt d es (Prompt p)) = State (Val (Prompt p)) 
                                               d es (Prompt $ p+1)

eval (State (Val v) d es q) = case d of
  Hole -> case es of
    (e:es') -> case e of
      (Val (Prompt p)) -> State (Val v) Hole es' q
      otherwise -> State (Val v) e es' q
    otherwise -> State (Val v) d es q
  otherwise -> State (ret d (Val v)) Hole es q

\end{Verbatim}


%TODO: show example output for a given term
%TODO: evaluation of different implementation approaches

%\chapter{Translations}

In this chapter, we develop a interpretation of \lmu\ in DCC.
We prove some properties of this interpretation, including \emph{soundness}.
We concatenate this interpretation with van Bakel's interpretation of \ltry\ in \lmu.
This concatenation yields an interpretation of \ltry\ in DCC.
This will then be used as a basis for the implementation of \ltry\ in Haskell.

\section{Interpreting \ltry\ in \lmu}

Steffen van Bakel describes the interpretation of \ltry\ to \lmu:

\[
  \begin{array}{rcl}
    \tr{x} &\triangleq& x \\
    \tr{`lx.M} &\triangleq& `lx.\tr{M} \\
    \tr{M N} &\triangleq& \tr{M}\tr{N} \\
    \multicolumn{3}{l}{\tr{\try M;\ \mcatch;\ \catch{m($x$) = $L$}}} \\
    & \triangleq & \\
    &\multicolumn{2}{l}{ (`lc_m.`m\text{m}.[\text{m}]\tr{\try M;\ \mcatch})(`lx.\tr{L})} \\
    
    \tr{\try M;\ \catch{m($x$) = $L$}} & \triangleq & (`lc_m.`m\text{m}.[\text{m}]\tr{M})(`lx.\tr{L}) \\
    \tr{\throw{n($M$)}} &\triangleq& `l\nonocc.[\text{n}]c_n\tr{M}
  \end{array}
\]

$\throw{n($M$)}$ terms are modelled using \lmu-abstractions of non-occurring names. This has the effect of removing all terms it is applied to:

\[
  (`m\nonocc.M)NOP \to (`m\nonocc.M)OP \to (`m\nonocc.M)P \to `m\nonocc.M
\]

The contents of the \lmu-abstraction calls $c_n$.
This \lam-variable is bound by the translation of \textbf{try} terms.
This binding means that the exception handlers, represented by $`lx.\tr{L}$,
are in scope for the reduction of the body of the try $M$.

% TODO: explain translation ?

\section{Interpreting \lmu\ in DCC}

% TODO: explain initial attempt and explain why it was simple and intuitive
%       and follow up with why it went wrong and what we had to change.
%       surprisingly did not work. replacing alpha in wsc terms to distribute
%       context throughout is why new one works

%\begin{figure}[!h]
%\definition{
%  \textsc{(Interpretation of \lmu\ into DCC)}
%  \[
%  \begin{array}{lcl}
%    \dbr{x}        & \triangleq & x \\
%    \dbr{`lx.M}    & \triangleq & `lx.\dbr{M} \\
%    \dbr{M N}      & \triangleq & \dbr{M} \dbr{N} \\
%    \dbr{`m`a.M}   & \triangleq & (`lp.\pp\ p\ \\
%    &&                 \hspace{1cm} ((\wsc\ p\ (`l`a.\dbr{M})) \\ 
%    &&                 \hspace{1cm} \dbr{N}) \\
%    &&               )\ \np \\
%    \dbr{[`b]M} & \triangleq & \psc\ `b\ \dbr{M}
%  \end{array}
%  \]
%}
%\end{figure}

The translation of \lmu-terms into DCC assumes that there is a single global prompt \gp. 
It also assumes that this prompt has already been pushed onto the stack.
This means that the translation of a full \lmu-program $M$ in DCC is:

\definition{
  \textsc{(Initialization of stack for running $M$ in DCC)}
  \[ (`l\gp.\pp\ \gp\ \dbr{M})\ \np \]
}
This creates a new prompt \gp\ which is in scope for all terms in $M$.
It also prepares the stack by pushing \gp\ immediately. 
With the stack prepared, 
the interpretation of \lmu\ terms into DCC proceeds as follows:

\definition{
  \textsc{(Interpretation of \lmu\ into DCC)}
  \[
  \begin{array}{lcl}
    \tr{x}        & \triangleq & x \\
    \tr{`lx.M}    & \triangleq & `lx.\tr{M} \\
    \tr{M N}      & \triangleq & \tr{M} \tr{N} \\
    \tr{`m`a.M}   & \triangleq & \wsc\ \gp\ `l`a.\pp\ \gp\ \tr{M} \\
    \tr{[`b]M} & \triangleq & \psc\ `b\ \tr{M}
  \end{array}
  \]
}

To implement \lmu-abstractions, we capture the subcontinuation until the last occurrence of \gp\ on the stack.
This subcontinuation is bound to $`a$ which ensures the subcontinuation is distributed to all occurrences of $`a$ in $M$.
\gp\ is then pushed back onto the stack before the evaluation of $M$.

To implement named-terms, the subcontinuation $`b$ is pushed into the stack before evaluating $M$.
This means the reduct of $M$ will be returned to this subcontinuation.
In effect, this reduces $M$ and passes the result to $`b$. 

\begin{example}{$\tr{`m`a.[`a](`lx.x)} \to \tr{`lx.x}$}
\[
\begin{array}{llrr}
             & \tr{`m`a.[`a](`lx.x)}  \\
  \triangleq & \wsc\ \gp\ `l`a.\pp\ \gp\ `lx.x, & \square, & \gp:[] \\
  \tob\      & (`l`a.\pp\ \gp\ `lx.x)(\square), & \square, & [] \\
  \tob\      & (\pp\ \gp\ `lx.x)[\square/`a],   & \square, & [] \\
  \tob\      & \pp\ \gp\ `lx.x,                 & \square, & [] \\
  \tob\      & `lx.x,                           & \square, & \gp:[]
\end{array}   
\]
The final state has restored the initial state of the stack by pushing \gp\ back on.
\end{example}


% TODO: Add proofs for completeness and other properties
\begin{theorem}[Soundness of $\tr{\bullet}$]
If $M \rightarrow_{`m} N$ then $\dbr{M} \rightarrow_{DCC} \dbr{N}$
\end{theorem}

% TODO: separate D and E in DCC proofs 
\begin{proof}{By induction on the definition of $\rightarrow_{`m}$}
\[
\begin{array}{rlrl}
  \prooflabel{(`lx.M)N \to M[N/x]:} \\
               & \tr{(`lx.M)N} \\
    \triangleq & \tr{(`lx.M)} \tr{N} \\
    \triangleq & (`lx.\tr{M}) \tr{N}, & \square, & \gp:[] \\
    \tob\   & \tr{M}[\tr{N}/x], & \square, & \gp:[] \\
    \triangleq\ & \tr{M[\tr{N}/x]}, & \square, & \gp:[] \\
\end{array} 
\]
\\
\[
\begin{array}{rlrr}
  \prooflabel{(`m`a.[`b]M)N \to `m`a.([`b]M)[[`a]M^\prime N/[`a]M^\prime]:} \\
             & \tr{(`m`a.[`b]M)N} \\
  \triangleq & \tr{(`m`a.[`b`]M)} \tr{N} \\
  \triangleq & (\wsc\ \gp\ `l`a. \pp\ \gp\ (\psc\ `b\ \tr{M})) \tr{N}, & \square, & \gp:[] \\
  \todcc\ & \wsc\ \gp\ `l`a. \pp\ \gp\ (\psc\ `b\ \tr{M}), & \square \tr{N}, & \gp:[] \\
  \todcc\ & (`l`a. \pp\ \gp\ (\psc\ `b\ \tr{M}))(\square\tr{N}), & \square, & [] \\
  \tob\ & (\pp\ \gp\ (\psc\ `b\ \tr{M}))[\square\tr{N}/`a], & \square, & [] \\
  \tob\ & \pp\ \gp\ (\psc\ `b\ (\tr{M}[\square\tr{N}/`a])), & \square, & [] \\
  \todcc\ & \psc\ `b\ (\tr{M}[\square\tr{N}/`a]), & \square, & \gp:[] \\
  \triangleq\ & \tr{`m`a.([`b]M)[[`a]M^\prime N/[`a]M^\prime]} \\
\end{array} 
\]
\\
\[
\begin{array}{rlrr}
  \prooflabel{`m`a.[`a]M \to M:} \\
              & \tr{`m`a.[`a]M} \\
   \triangleq & \wsc\ \gp\ `l`a.\pp\ \gp\ (\psc\ `a\ \tr{M}), & \square, & \gp:[]  \\
   \todcc\    & `l`a.\pp\ \gp\ (\psc\ `a\ \tr{M})(\square), & \square, & [] \\
   \tob\      & \pp\ \gp\ (\psc\ `a\ \tr{M})[\square/`a], & \square, & [] \\
   \tob\      & \pp\ \gp\ (\psc\ \square\ (\tr{M}[\square/`a]), & \square, & [] \\
   \todcc\    & \psc\ \square\ (\tr{M}[\square/`a], & \square, & \gp:[] \\
   \todcc\    & \tr{M}[\square/`a], & \square,  & \gp:[] \\
   \triangleq & \tr{M} \\
\end{array}
\]
\\ 
\[
\begin{array}{rlrr}
  \prooflabel{`m`a.[`b]`m`g.[`d]M \to `m`a.[`d](M[`b/`g]):} \\
    \triangleq & \wsc\ \gp\ `l`a.\pp\ \gp\ \tr{[`b]`m`g.[`d]M}, & \square, & \gp:[] \\
    \todcc\    & `l`a.\pp\ \gp\ \tr{[`b]`m`g.[`d]M}(\square), & \square, & [] \\
    \tob\      & (\pp\ \gp\ \tr{[`b]`m`g.[`d]M})[\square/`a], & \square,& [] \\
    \tob\      & \pp\ \gp\ \tr{[`b]`m`g.[`d]M}[\square/`a], & \square,& [] \\
    \tob\      & \tr{[`b]`m`g.[`d]M}[\square/`a], & \square,& \gp:[] \\
    \triangleq & (psc `b \tr{`m`g.[`d]M})[\square/`a], & \square,& \gp:[] \\
    \triangleq & \tr{`m`g.[`d]M})[\square/`a], & \square, & `b:\gp:[] \\
    \triangleq & (\wsc\ \gp\ `l`g.\tr{[`d]M})[\square/`a], & \square,& `b:\gp:[] \\
    \triangleq & \wsc\ \gp\ `l`g.\pp\ \gp\ \tr{[`d]M}[\square/`a], & \square, & `b:\gp:[] \\
    \triangleq & (`l`g.\pp\ \gp\ \tr{[`d]M}[\square/`a])(`b), & \square, & [] \\
    \triangleq & (\pp\ \gp\ \tr{[`d]M}[\square/`a])[`b/`g], & \square,& [] \\
    \triangleq & \pp\ \gp\ (\tr{[`d]M}[\square/`a])[`b/`g], & \square,& [] \\
    \triangleq & \tr{[`d]M}[\square/`a])[`b/`g], & \square,& \gp:[] \\
    \triangleq & (\psc\ `d\ M[\square/`a][`b/`g], & \square,& \gp:[] \\
    \triangleq & \psc\ `d\ (M[\square/`a])[`b/`g], & \square,& \gp:[] \\
    \triangleq & \tr{`m`a.[`d](M[`b/`g])}, & \square,& \gp:[] \\
\end{array}
\]
\\ 
\[
\begin{array}{rlrl}
  \prooflabel{`m`a.[`b]`m`g.[`g]M \to `m`a.[`b](M[`b/`g]):} \\
               & \tr{`m`a.[`b]`m`g.[`d]M} \\
    \triangleq & \wsc\ \gp\ `l`a.\pp\ \gp\ \tr{[`b]`m`g.[`d]M},    & \square, & \gp:[] \\
    \todcc\    & `l`a.\pp\ \gp\ \tr{[`b]`m`g.[`d]M}(\square),    & \square, & [] \\
    \tob\      & (\pp\ \gp\ \tr{[`b]`m`g.[`d]M})[\square/`a],    & \square, & [] \\
    \tob\      & \pp\ \gp\ \tr{[`b]`m`g.[`d]M}[\square/`a],      & \square, & [] \\
    \tob\      & \tr{[`b]`m`g.[`d]M}[\square/`a],              & \square, & \gp:[] \\
    \triangleq & (psc `b \tr{`m`g.[`d]M})[\square/`a],         & \square, & \gp:[] \\
    \triangleq & \tr{`m`g.[`d]M})[\square/`a],                 & \square, & `b:\gp:[] \\
    \triangleq & (\wsc\ \gp\ `l`g.\tr{[`d]M})[\square/`a],       & \square, & `b:\gp:[] \\
    \triangleq & \wsc\ \gp\ `l`g.\pp\ \gp\ \tr{[`d]M}[\square/`a], & \square, & `b:\gp:[] \\
    \triangleq & (`l`g.\pp\ \gp\ \tr{[`d]M}[\square/`a])(`b),    & \square, & [] \\
    \triangleq & (\pp\ \gp\ \tr{[`d]M}[\square/`a])[`b/`g],      & \square, & [] \\
    \triangleq & \pp\ \gp\ (\tr{[`d]M}[\square/`a])[`b/`g],      & \square, & [] \\
    \triangleq & \tr{[`d]M}[\square/`a])[`b/`g],               & \square, & \gp:[] \\
    \triangleq & (\psc\ `d\ M[\square/`a][`b/`g],              & \square, & \gp:[] \\
    \triangleq & \psc\ `b\ (M[\square/`a])[`b/`g],             & \square, & \gp:[] \\
    \triangleq & \tr{`m`a.[`b](M[`b/`g])},                     & \square, & \gp:[]
\end{array}
\]
\\ 
% TODO: verify that this is correct
\[
\begin{array}{rlrl}
  \prooflabel{(`m`d.[`a]M)[[`a]M^\prime N/[`a]M^\prime]
    \to (`m`d.[`a](M[[`a]M^\prime N/[`a]M^\prime])N} \\
               & \tr{(`m`d.[`a]M)[[`g]M^\prime N/[`a]M^\prime]}\\
    \triangleq & (\wsc\ \gp\ `l`d.\pp\ \gp\ (\psc\ `a\ M))[\square N/`a]         & \square, & \gp:[]  \\
    \tob\      & \wsc\ \gp\ `l`d.\pp\ \gp\ (\psc\ \square N\ (M[\square N/`a]))) & \square, & \gp:[]  \\
    \triangleq & \tr{`m`d.[`a](M[[`a]M^\prime N/[`a]M^\prime])N}
\end{array}
\]
\\ 
% TODO: verify that this is correct
\[
\begin{array}{rlrl}
  \prooflabel{M[[`a]M^\prime N/[`a]M] \to M \hspace{10pt} (`a \not\in \fn(M)):} \\
               & \tr{M[[`a]M^\prime N/[`a]M]} \\
    \triangleq & \tr{M}[\square N/`a] \\
    \tob\      & \tr{M} & (`a \not\in \fv(M))
\end{array}
\]
\end{proof}

This means that the translation of $M$ reduces a term $Q$ that relates to the untranslated reduct of $N$, given $M \to N$.
The relation is either that $Q$ reduces to $N$, is the same as $N$, or $N$ reduces to $Q$.

\begin{theorem}[Completeness of $\tr{\bullet}$]

\end{theorem}

We attempt to prove one of the following properties:
\begin{enumerate}
  \item $\tr{M} \to Q \Rightarrow \exists N. M \to N \land Q \to^* \tr{N}$
  \item $\tr{M} \to^{nf} Q \Rightarrow \exists N. M \to^* N \land \tr{N} = Q$
  \item $\tr{M} \to^{nf} Q \Rightarrow \exists N. M \to^* N \land \tr{N} \to^{nf} Q$
\end{enumerate}

\section{Interpreting \ltry\ in DCC}

  By appending the interpretation of \ltry\ in $`l`m$ with the interpretation of
  $`l`m$ in DCC, we get a translation from \ltry\ to DCC:

  % TODO: add lines over (catch n(x) = L)
  \definition{
    \textsc{Translation of \ltry\ into DCC}
    \[
    \begin{array}{rcl}
      \tr{x} & \triangleq & x \\
      \tr{`lx.M} & \triangleq & `lx.\tr{M} \\
      \tr{MN} & \triangleq & \tr{M} \tr{N} \\
      
      \tr{\throw{n($M$)}} & \triangleq & \wsc\ \gp\ `l\nonocc.\pp\ \gp\ (psc\ n\ (c_n\ \tr{M})) \\ 
      \tr{\try M;\ \catch{n($x$) = $L$}}
        &
        \triangleq 
        &
        (`lc_n.\wsc\ \gp\ `ln.\pp\ \gp\ (\psc\ n\ \tr{M}))(`lx.\tr{L})
      \\ 
      %\multicolumn{3}{l}{\tr{\try M;\ \catch{n$_i$($x$) = $M_i$};\ \catch{m($x$) = $L$}}} \\
      \multicolumn{3}{l}{\tr{\try M;\ \mcatch;\ \catch{m($x$) = $L$}}} \\
        & \triangleq \\
      \multicolumn{3}{r}{
        %(`lc.\wsc\ \gp\ `lm.\pp\ \gp\ (\psc\ m\ \tr{\try M;\ \catch{n$_i$($x$) = $M_i$}}))(`lx.\tr{L})
        (`lc_m.\wsc\ \gp\ `lm.\pp\ \gp\ (\psc\ m\ \tr{\try M;\ \mcatch}))(`lx.\tr{L})
      }
    \end{array} 
    \]
  }

% TODO: describe the interpretation

%include{implementation}
%\chapter{Conclusion}

This chapter evaluates parts of the methodology and findings of the project.
With specific reference to individual components, it examines what should have been done differently.
It closes with a presentation of suitable directions for future work, 
especially regarding research questions thrown up and still unanswered by the present work.

\subsection{Evaluation}
\subsubsection{Interpreter}
We often treat terms generically: we are not specifically interested in the form of the term.
For example the $M$ and $N$ in the term $M[N/x]$ are generic.
Generic terms like this were used extensively in the soundness and completeness proofs for the \lmu\ translation.
A minor extension to the interpreter would be to extend it with the constants.
Like a value, the evaluation of a constant would leave the constant the same. 
This extension would allow the expression of these generic terms.
In this way, the output of the interpreter could have been used to produce proofs without requiring amendments.
Additionally, 
the interpreter could have output \LaTeX\ by writing an appropriate function of type \mono{State -> String} to pass into the evaluation engine.

\subsubsection{Choice of Calculus}
CDC was not the correct choice of calculus to facilitate a translation from \ltry\ to Haskell.  
It was chosen because it easily facilitated a translation to Haskell: 
the syntax of CDC was intentionally close to Haskell
and a CDC library had been implemented in Haskell by the authors of the calculus.
However there is an impedance mismatch between CDC and \lmu.
The translation does not use most of the expressive power of CDC.
For example, 
CDC is capable of generating multiple prompts of different names.
Only a single prompt is used by the translation of the \lmu-calculus. 
We do not need a continuation stack or multiple named prompts, we just need a single delimiter.
We could get away with a map from prompts to contexts in place of a stack.
This would have greatly simplified the translation although it would also require reimplementing the CDC operators.
We believe this would be useful to develop in follow up work.

\subsubsection{Direct Translation}
There is possible a more direct translation of \ltry\ to CDC that is not mediated by a translation to \lmu.
More importantly, there is likely to be a more direct implementation of \ltry\ to Haskell.
By translating through two other calculi, we have left this possibility unexamined.
Although we still would want a direct implementation of \ltry\ in Haskell that does not use CDC,
we have demonstrated that an implementation is possible and laid down the ground work for this to be done.

%\subsubsection{Types}
%We did not investigate whether types were preserved by the translation.
%
%The typing of CDC terms is left to Haskell.
%A formal treatment would have been better.

\subsubsection{Translation and Soundness}
The embedding of the \lmu-calculus in CDC is not as close to the \lmu-calculus as it could be.
There are some reduction steps in the destination language that are not reflected by the source language.
Specifically, the soundness property we proved was
\[
  M \to_{`m} N \Rightarrow \exists P. \tr{M} \to^{*} P \land \tr{N} \to^{*} P
\]
This says that the translation of $M$ and the translation of the $`m$-reduction of $M$ reduce to the same term.
By defining the substitition of terms explicitly, as in \cite{Bakel14}, we may have proved
\[
  M \to_{`m}^{*} N \Rightarrow \tr{M} \to^{*} \tr{N}
\]
This says that the translation of $M$ reduces to the translation of $N$.
With this property, the correspondence between the source language and the target language would have be closer.

\subsection{Conclusion}
We defined an interpreter for CDC which we used to experiment with translations from the \lmu-calculus.
The output of the interpreter was transliterated into derivations that we used to prove the soundness and completeness of our translation.
Using this translation, along with van Bakel's \ltry-to-\lmu\ translation, we defined a translation from \ltry\ to CDC.
This was used as the basis for some proof-of-concept \ltry\ implementations in Haskell.
This project presents a clear demonstration that named exceptions, as introduced by the \ltry-calculus, are implementable in Haskell.
Apart from a minor syntactic extension,
the implementation would not require any machinery currently unavailable to Haskell.
    
\subsubsection{Future Work}
Finally, we present a list of suitable directions for future work.
\begin{itemize}
  \item \emph{Type preservation} -- Both \ltry\ and \lmu\ have type assignment systems. We did not explore whether the types are preserved under the translations we defined. Doing this is will give additional insight into how the \ltry\ calculus relates to delimited continuations and how it can be implemented in Haskell without using CDC.
  \item \emph{Haskell extension} -- Proof-of-concept Haskell libraries are presented and a language extension is proposed but the language extension is not implemented. Given the state of the present work, this will be trivial. By exposing the functionality in a language pragma, we will extend the parser with additional grammar rules.
  \item \emph{Reimplement CDC without stack} -- The implementation of \lmu\ in CDC only ever uses a single prompt: CDC is overengineered for the purposes of a \lmu-translation. We will rewrite CDC to replace the context and prompt stack with a mapping from prompts to contexts. Using this, a simpler translation of \lmu\ to CDC should be found.
  \item \emph{Translate \ltry\ directly into Haskell} -- The current methodology helped expand our understanding of the relation of \ltry\ to Haskell. Using this, we can write a direct implementation of \ltry\ into Haskell that bypasses CDC entirely.
\end{itemize}

%
\newpage
\bibliographystyle{plain}
\bibliography{references}

\end{document}
